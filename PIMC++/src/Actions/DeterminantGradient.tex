\documentclass{article}
\usepackage{delarray}
\usepackage{amsmath}
\title{Gradients of determinant wave functions}
\author{Kenneth P. Esler Jr.}
\date{\today}

\newenvironment{DMatrix}{\begin{array}|{*{20}{c}}|}{\end{array}}
\newenvironment{Matrix}{\begin{array}[{*{20}{c}}]}{\end{array}}
\begin{document}
\maketitle

\section{Problem}
Consider a wave function of the form
\begin{equation}
  \psi(r_0  \dots r_N) = 
\begin{DMatrix}
\phi_0(r_0) & \phi_1(r_0) & \phi_2(r_0) & \dots  & \phi_N(r_0) \\
\phi_0(r_1) & \phi_1(r_1) & \phi_2(r_1) & \dots  & \phi_2(r_1) \\
\phi_0(r_2) & \phi_1(r_2) & \phi_2(r_2) & \dots  & \phi_2(r_2) \\
\vdots      & \vdots      & \vdots      & \ddots &  \vdots     \\
\phi_0(r_N) & \phi_1(r_N) & \phi_2(r_N) & \dots  & \phi_N(r_N)
\end{DMatrix}
\equiv \det A
\end{equation}
Recall that in the Laplace expansion, a determinant can be written as
a sum over any row or column of the matrix element times its cofactor,
\begin{equation}
\det A = \sum_i A_{ij} (\text{cof}\ A)_{ij}.
\end{equation}
In the Laplace expansion, the cofactor of a given element $A_{ij}$ is
given by $(-1)^{i+j}$ times the determinant of the matrix constructed
by removing the $i^\text{th}$ row and $j^\text{th}$ column from $A$.
It is clear, then, that if we use the cofactor expansion along the
$n^\text{th}$ row, the cofactors will not depend on $r_n$.  This
greatly simplifies the gradient calculation since no chain rule will
come in.  

Unfortunately, the Laplace expansion, while illustrative, is
computationally extremely costly.  Fortunately, the reader may 
recall that the cofactor matrix can be given by
\begin{equation}
\text{cof}\ A = \det(A) \left(A^{-1}\right)^T.
\end{equation}
The determinant may be computed by $LU$ factorization in
$\mathcal{O}(N^3)$ operations.  Let us then write the tensor
formed by taking the gradients of the elements of $A$,
\begin{equation}
\mathbf{G}_{ij} \equiv \nabla \phi_j (r_i)
\end{equation}
Then, we can write
\begin{eqnarray}
\nabla_{r_j} \psi(r_0 \dots r_N) & = &
\sum_i G_{ij} (\text{cof}\  A)_{ij} \\
& = & \sum_i \det(A) \ G_{ij} \left( A^{-1}\right)_{ji}
\end{eqnarray}
Often, the quantity we require is $\nabla_{r_j} \ln(\psi)$, which is
then given by
\begin{equation}
\nabla_{r_j} \ln\left[\psi(r_0 \dots r_N)\right] =
\sum_i G_{ij} \left( A^{-1}\right)_{ji}
\end{equation}


\end{document}
