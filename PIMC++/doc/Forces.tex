\documentclass{article}
\usepackage{amsmath}
\usepackage{cancel}
\title{Force calculations with PIMC}
\author{Kenneth P. Esler Jr.}
\date{\today}

\newcommand{\vr}{\mathbf{r}}
\newcommand{\vrp}{\mathbf{r}'}
\newcommand{\vR}{\mathbf{R}}
\newcommand{\vF}{\mathbf{F}}

\begin{document}
\maketitle

\section{Introduction}
In this chapter, we address the specifics of how thermal forces on
classical ions may be computed within the PIMC framework.  We begin by
recalling from statistical mechanics that the average force exerted in
particle $i$ in the canonical ensemble may be given by
\begin{equation}
\vF_i = -\nabla_{\vR_i} \mathcal{F},
\end{equation}
where $\mathcal{F}$ is the Helmholtz free energy.  Then we may write 
\begin{eqnarray}
\vF_i & = & \frac{1}{\beta}  \nabla_{\vR_i} \ln Z \\
& = & -\frac{1}{\beta}
\left\langle 
\nabla_{\vR_i} S
\right\rangle_{\text{PIMC}},
\end{eqnarray}
where $S$ denotes the total action, and the average is over all paths
in the PIMC simulation.

\section{Kinetic action gradients}
Clearly, the kinetic action of the electrons has no explicit dependence
on the location of the ions.  Hence, the gradient of this term is
identically zero.

\section{Pair action gradients}

\begin{eqnarray}
\vr \equiv   \vr_\text{elec} - \vR \\
\vrp \equiv  \vrp_\text{elec} - \vR
\end{eqnarray}
Now $u = u(\vr,\vrp;\tau)$.  We wish to compute the gradient,
$\nabla_\vR u(\vr, \vrp; \tau)$.  Because of symmetry, it is
convenient to store, $u(\vr, \vrp;\tau) = u(q,s,z)$, where
\begin{eqnarray}
q \equiv \frac{|\vr| + |\vrp|}{2} \\
z \equiv |\vr| - |\vrp| \\
s \equiv |\vr - \vrp|
\end{eqnarray}
Since $s$ is independent of $\vR$, we may write
\begin{equation}
\nabla_\vR u = \frac{\partial u}{\partial q} \nabla_\vR q +
\frac{\partial u}{\partial z} \nabla_\vR z
\end{equation}
Then we may write
\begin{eqnarray}
& & \\
q & = & \frac{1}{2} \left\{\left[ (R_x - r_x)^2 + (R_y - r_y)^2 +
  (R_z-r_z)^2\right]^{\frac{1}{2}}+ 
\left[(R_x-r'_x)^2 +(R_y-r'_y)^2 +
  (R_z-r'_z)^2\right]^\frac{1}{2}\right\} \nonumber \\
z & = & \rule{0.26cm}{0.0cm} \left\{\left[ (R_x - r_x)^2 + (R_y - r_y)^2 +
  (R_z-r_z)^2\right]^{\frac{1}{2}}- 
\left[(R_x-r'_x)^2 +(R_y-r'_y)^2 +
  (R_z-r'_z)^2\right]^\frac{1}{2}\right\} \nonumber 
\end{eqnarray}
Hence,
\begin{eqnarray}
  \nabla_\vR q & = & -\frac{\hat{\vr} + \hat{\vr}'}{2} \\
  \nabla_\vR z & = & -\left[\hat{\vr} - \hat{\vr}'\right] 
\end{eqnarray}
Then, we have
\begin{equation}
\nabla_{\vR} u(q,z,s;\tau) = -\left[\frac{1}{2}\frac{\partial u}{\partial q}
\left(\hat{\vr} + \hat{\vr}'\right)
-\frac{\partial u}{\partial z} \left(\hat{\vr} -\hat{\vr}'\right) \right]
\end{equation}

\subsection{Tricubic Splines}
A very general way to store the pair action, $u(q,z,s;\tau)$, is the
use of tricubic spline interpolation.  This method tabulates the
values of the action on a $3D$ mesh, and uses piecewise tricubic
interpolants constructed such that the value and all first and second
derivatives are continuous everywhere.  The use of these splines
requires an box-shaped domain.  Because of their definitions, the
range of $z$ depends on $q$ and the range of $s$ depends on both $q$
and $z$.  Thus, we must choose an alternate set of variables such that
the spline has a box-shaped domain in those variables.  For reasons of
accuracy, we choose,
\begin{eqnarray}
q & \equiv & q \\
y & \equiv & \frac{|z|}{z_\text{max}} \\
t & \equiv & \frac{s-|z|}{z_\text{max}-|z|},
\end{eqnarray}
where
\begin{equation}
z_\text{max} = \min \left[2q, z^*_\text{max}(\tau)\right].
\end{equation}
$z^*_\text{max}(\tau)$ is an appropriately chosen constant to reflect
the range of $z$ likely to occur in PIMC simulation.  On the rare
occasion that $z$ exceeds $z^*_\text{max}$, we extrapolate from the
last point using derivative information from the spline.

Because $z_\text{max}$ may depend on $q$, we must compute the partial
derivatives of $u$ carefully.
\begin{eqnarray}
\left.\frac{\partial u}{\partial q}\right|_{z,s} & = & 
\left.\frac{\partial u}{\partial q}\right|_{y,t} \ \ \ \ \, + 
\left.\frac{\partial u}{\partial y}\right|_{q,t} \frac{\partial y}{\partial q}+
\left.\frac{\partial u}{\partial t}\right|_{q,y} \frac{\partial t}{\partial q} \\
\left.\frac{\partial u}{\partial z}\right|_{q,s} & = & 
\left.\frac{\partial u}{\partial q}\right|_{y,t} \cancelto{0}{\frac{\partial q}{\partial z}}+
\left.\frac{\partial u}{\partial y}\right|_{q,t} \frac{\partial y}{\partial z}+
\left.\frac{\partial u}{\partial t}\right|_{q,y} \frac{\partial t}{\partial z} \\
\left.\frac{\partial u}{\partial s}\right|_{q,z} & = & 
\left.\frac{\partial u}{\partial q}\right|_{y,t} \cancelto{0}{\frac{\partial q}{\partial s}}+
\left.\frac{\partial u}{\partial y}\right|_{q,t} \frac{\partial y}{\partial s}+
\left.\frac{\partial u}{\partial t}\right|_{q,y} \frac{\partial t}{\partial s} \\
\end{eqnarray}

\section{Long-range forces}
We recall the expressions for the long-range contribution to the
action at a given timeslice,
\newcommand{\vk}{\mathbf{k}}
\begin{eqnarray}
U_\text{long} & = & \sum_{\alpha > \beta} \left[ \sum_\vk \mathcal{R}e\left(\rho_\vk^\alpha
  \rho_{-\vk}^\beta\right) u^{\alpha \beta}_k   -N^\alpha N^\beta
  u^{\alpha \beta}_{s\mathbf{0}}  +\tilde{U}_{k=0} \right] \\
& + & \sum_\alpha \left[ -\frac{N^\alpha u_l^{\alpha \alpha}(0)}{2} + \frac{1}{2} \sum_\vk |\rho_\vk^\alpha|^2 u^{\alpha\alpha}_\vk - \frac{1}{2}\left(N_\alpha\right)^2
  u_{s\mathbf{0}}^{\alpha\alpha} +\tilde{U}_{k=0}\right]  \nonumber 
\end{eqnarray}
We note that the only terms which depend on the ion positions, $\vR$,
are the $\rho_\vk$.  Let $\alpha$ represent the ion species.  Then $\rho^\alpha$ is given by \begin{eqnarray}
\rho^\alpha_\vk & = & \sum_j e^{i\vk \cdot \vR_j} \\
\nabla_{\vR_j} \rho_\vk^\alpha & = & i\vk
\end{eqnarray}
\begin{eqnarray}
\nabla_{\vR_j} U_\text{long} & = &
-\sum_\vk \vk \left[ \mathcal{I}m \left(\rule{0.0cm}{0.35cm}e^{i\vk\cdot\vR_j}\rho_{-\vk}^\alpha\right)u^{\alpha
    \alpha}_k +\sum_{\beta \neq \alpha} \mathcal{I}m\left(
e^{i\vk\cdot\vR_j}\rho^\beta_{-\vk}\right) u_k^{\alpha\beta} \right] \\
& = & -\sum_\vk \vk \sum_\beta \mathcal{I}m
\left(e^{i\vk\cdot\vR_j}\rho_{-\vk}^\beta\right) u^{\alpha\beta}_k \\
& = & - \sum_\vk \vk \ \mathcal{I}m \left( e^{i\vk\cdot\vR_j} \sum_\beta
\rho_{-\vk}^\beta u^{\alpha\beta}_k \right)
\end{eqnarray}

\subsection{Storage issues}
We reduce our calculation of $\rho^{\alpha\beta}_\vk$ by noting that
$\rho_{-\vk}^{\alpha \beta} = \left( \rho_\vk^{\alpha
  \beta}\right)^*$, and storing only half of the $\vk$-vectors.
Therefore, we need to compensate for this storage by explicitly
considering the vectors in $(\vk,-\vk)$ pairs.  We may then write
\begin{eqnarray}
\nabla_{\vR_j} U_\text{long} & = & 
-\sum_{\text{sgn}(\vk)\ge 0} \left[\vk \mathcal{I}m 
\left( e^{i\vk\cdot\vR_j} \sum_\beta \rho_{-\vk}^\beta
u_k^{\alpha\beta}\right) - \vk \mathcal{I}m 
\left(
e^{-i\vk\cdot\vR_j} \sum_\beta \rho_{\vk}^\beta
u_k^{\alpha\beta}\right) 
\right] \nonumber \\
& = & -2 \sum_{\text{sgn}(\vk)\ge 0} \vk \ \mathcal{I}m
\left( e^{i\vk\cdot\vR_j} \sum_\beta \rho^\beta_{-\vk}
u_k^{\alpha\beta} \right)
\end{eqnarray}

\section{Nodal action}
If, in our restricted path integral formulation for fermions, our
nodal surfaces are parameterized by the positions of the
ions$\{\vR_i\}$, there must be a contribution to the gradient of the
action, $\nabla_{\vR_i} S$, for a given instantaneous path in the
simulation.  In this section, however, we argue that if our nodal
surfaces correspond to those of the exact density matrix, the net
contribution to the force when average over all paths will be zero.
In practice, except for trivial problems, we never have exact nodal
surfaces.  However, the approximation made in neglecting the
contribution due to inexact nodes is no worse than the approximation
of using the restricted path approach to begin with.

To begin our argument, we return to partition function.  The partition
function may be written as
\begin{equation}
Z = \sum_{s\in \text{states}} e^{-\beta E_s}
\end{equation}
Then
\begin{eqnarray}
\vF_i & = & \frac{1}{\beta Z} \nabla_{\vR_i} Z \\
      & = & \frac{1}{\beta Z} \sum_s \nabla_{\vR_i} e^{-\beta E_s} \\
      & = & \frac{1}{\beta Z} \sum_s \left\langle \psi_s |
      \nabla_{\vR_i} V | \psi_s \right \rangle e^{-\beta E_s}.
\end{eqnarray}
The last is due to the Hellmann-Feynman.  
Since $\nabla_{\vR_i} V$ is local, the force can clearly be written in
terms of an integral over the thermal electron density.  Thus, there
must clearly be an estimator for $\vF_i$ which does not have an
explicit dependence on the nodal structure.  This structure comes in
only in specifying the boundary conditions on the $|\psi_s\rangle$s
that determine the charge density.

The estimator we have been using above is not the direct
Hellmann-Feynman estimator, but one base on pair actions.



\end{document}
