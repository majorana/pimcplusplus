\documentclass{article}
\usepackage{amsmath}
\usepackage{delarray}
\author{Kenneth P. Esler Jr.}
\date{\today}
\title{Cubic Spline Interpolation in 1, 2 and 3 Dimensions}

\newenvironment{DMatrix}{\begin{array}|{*{20}{c}}|}{\end{array}}
\newenvironment{MyMatrix}{\begin{array}({*{20}{c}})}{\end{array}}
\newenvironment{LMatrix}{\begin{array}({*{20}{l}})}{\end{array}}

\begin{document}
\maketitle
\abstract{We present the basic equations and algorithms necessary to
  construct and evaluate cubic interpolating splines in one, two, and
  three dimensions.  Equations are provided for both natural and
  periodic boundary conditions.}

\section{One Dimension}
Let us consider the problem in which we have a function $y(x)$
specified at a discrete set of points $x_i$, such that $y(x_i) = y_i$.
We wish to construct a piece-wise cubic polynomial interpolating
function, $f(x)$, which satisfies the following conditions:
\begin{itemize}
\item $f(x_i) = y_i$
\item $f'(x_i^-) = f'(x_i^+)$
\item $f''(x_i^-) = f''(x_i+)$
\end{itemize}

\subsection{Hermite Interpolants}
In our piecewise representation, we wish to store only the values,
$y_i$, and first derivatives, $y'_i$, of our function at each point
$x_i$, which we call {\em knots}.  Given this data, we wish to
construct the piecewise cubic function to use between $x_i$ and
$x_{i+1}$ which satisfies the above conditions.  In particular, we
wish to find the unique cubic polynomial, $P(x)$ satisfying
\begin{eqnarray}
P(x_i)      & = & y_i      \label{eq:c1} \\
P(x_{i+1})  & = & y_{i+1}  \label{eq:c2} \\
P'(x_i)     & = & y'_i     \label{eq:c3} \\
P'(x_{i+1}) & = & y'_{i+1} \label{eq:c4}
\end{eqnarray}
\begin{eqnarray}
h_i & \equiv & x_{i+1} - {x_i} \\
t & \equiv & \frac{x-x_i}{h_i}.
\end{eqnarray}
We then define the basis functions,
\begin{eqnarray}
p_1(t) & = & (1+2t)(t-1)^2  \label{eq:p1}\\
q_1(t) & = & t (t-1)^2      \\
p_2(t) & = & t^2(3-2t)      \\
q_2(t) & = & t^2(t-1)      \label{eq:q2}
\end{eqnarray}
On the interval, $(x_i, x_{i+1}]$, we define the interpolating
function,
\begin{equation}
P(x) = y_i p_1(t) + y_{i+1}p_2(t) + h\left[y'_i q_1(t) + y'_{i+1} q_2(t)\right]
\end{equation}
It can be easily verified that $P(x)$ satisfies conditions (\ref{eq:c1})
through (\ref{eq:c4}).  It is now left to
determine the proper values for the $y'_i\,$s such that the continuity
conditions given above are satisfied.

By construction, the value of the function and derivative will match
at the knots, i.e.
\begin{equation}
P(x_i^-) = P(x_i^+), \ \ \ \ P'(x_i^-) = P'(x_i^+).
\end{equation}
Then we must now enforce only the second derivative continuity:
\begin{eqnarray}
P''(x_i^-) & = & P''(x_i^+) \\
\frac{1}{h_{i-1}^2}\left[\rule{0pt}{0.3cm}6 y_{i-1} -6 y_i + h_{i-1}\left(2 y'_{i-1} +4 y'_i\right) \right]& = &
\frac{1}{h_i^2}\left[\rule{0pt}{0.3cm}-6 y_i + 6 y_{i+1} +h_i\left( -4 y'_i -2 y'_{i+1} \right)\right] \nonumber
\end{eqnarray}
Let us define
\begin{eqnarray}
\lambda_i & \equiv & \frac{h_i}{2(h_i+h_{i-1})} \\
\mu_i & \equiv & \frac{h_{i-1}}{2(h_i+h_{i-1})}  = \frac{1}{2} - \lambda_i.
\end{eqnarray}
Then we may rearrage,
\begin{equation}
\lambda_i y'_{i-1} + y'_i + \mu_i y'_{i+1} = \underbrace{3 \left[\lambda_i \frac{y_i - y_{i-1}}{h_{i-1}} + \mu_i \frac{y_{i+1}
    - y_i}{h_i} \right] }_{d_i}
\end{equation}
This equation holds for all $0<i<(N-1)$, so we have a tridiagonal set of
equations.  The equations for $i=0$ and $i=N-1$ depend on the boundary
conditions we are using.  
\subsection{Periodic boundary conditions}
For periodic boundary conditions, we have
\begin{equation}
\begin{matrix}
y'_0           & +  & \mu_0 y'_1     &   &                   &            & \dots                   & +  \lambda_0 y'_{N-1} & = & d_0 \\
\lambda_1 y'_0 & +  & y'_1           & + &  \mu_1 y'_2       &            & \dots                   &                       & = & d_1 \\
               &    & \lambda_2 y'_1 & + &  y'_2           + & \mu_2 y'_3 & \dots                   &                       & = & d_2 \\
               &    &                &   &  \vdots           &            &                         &                       &   &     \\
\mu_{N-1} y'_0 &    &                &   &                   &            & +\lambda_{N-1} y'_{N-1} & +  y'_{N-2}           & = & d_3 
\end{matrix}
\end{equation}
Or, in matrix form, we have,
\begin{equation}
\begin{MyMatrix}
1         & \mu_0     &    0   &   0           & \dots         &      0        & \lambda_0 \\
\lambda_1 &  1        & \mu_1  &   0           & \dots         &      0        &     0     \\
0         & \lambda_2 &   1    & \mu_2         & \dots         &      0        &     0     \\
\vdots    & \vdots    & \vdots & \vdots        & \ddots        &   \vdots      &  \vdots   \\
0         &   0       &   0    & \lambda_{N-3} &      1        & \mu_{N-3}     &    0      \\
0         &   0       &   0    &   0           & \lambda_{N-2} &      1        & \mu_{N-2} \\
\mu_{N-1} &   0       &   0    &   0           &   0           & \lambda_{N-1} &  1     
\end{MyMatrix}
\begin{MyMatrix} y'_0 \\ y'_1 \\ y'_2 \\ \vdots \\ y'_{N-3} \\ y'_{N-2} \\ y'_{N-1} \end{MyMatrix} =
\begin{MyMatrix} d_0  \\  d_1 \\  d_2 \\ \vdots \\  d_{N-3} \\  d_{N-2} \\  d_{N-1} \end{MyMatrix} .
\end{equation}
The system is tridiagonal except for the two elements in the upper
right and lower left corners.  These terms complicate the solution a
bit, although it can still be done in $\mathcal{O}(N)$ time.  We first
proceed down the rows, eliminating the the first non-zero term in each
row by subtracting the appropriate multiple of the previous row.  At
the same time, we also eliminate the first element in the last row,
shifting the position of the first non-zero element to the right with
each iteration.  When we get to the final row, we will have the value
for $y'_{N-1}$.  We can then procede back upward, backstubstituting
values from the rows below to calculate all the derivatives.

\subsection{Complete boundary conditions}
If we specify the first derivatives of our function at the end points,
we have what is known as {\em complete} boundary conditions.  The
equations in that case are trivial to solve:
\begin{equation}
\begin{MyMatrix}
1         &  0        &    0   &   0           & \dots         &      0        &     0     \\
\lambda_1 &  1        & \mu_1  &   0           & \dots         &      0        &     0     \\
0         & \lambda_2 &   1    & \mu_2         & \dots         &      0        &     0     \\
\vdots    & \vdots    & \vdots & \vdots        & \ddots        &   \vdots      &  \vdots   \\
0         &   0       &   0    & \lambda_{N-3} &      1        & \mu_{N-3}     &    0      \\
0         &   0       &   0    &   0           & \lambda_{N-2} &      1        & \mu_{N-2} \\
0         &   0       &   0    &   0           &   0           &      0        &  1     
\end{MyMatrix}
\begin{MyMatrix} y'_0 \\ y'_1 \\ y'_2 \\ \vdots \\ y'_{N-3} \\ y'_{N-2} \\ y'_{N-1} \end{MyMatrix} =
\begin{MyMatrix} d_0  \\  d_1 \\  d_2 \\ \vdots \\  d_{N-3} \\  d_{N-2} \\  d_{N-1} \end{MyMatrix} .
\end{equation}
This system is completely tridiagonal and we may solve trivially by
performing row eliminations downward, then proceeding upward as
before.

\subsection{Natural boundary conditions}
If we do not have information about the derivatives at the boundary
conditions, we may construct a {\em natural spline}, which assumes the
the second derivatives are zero at the end points of our spline.  In
this case our system of equations is the following:
\begin{equation}
\begin{MyMatrix}
1         & \frac{1}{2} &    0   &   0           & \dots         &      0        &     0     \\
\lambda_1 &  1          & \mu_1  &   0           & \dots         &      0        &     0     \\
0         & \lambda_2   &   1    & \mu_2         & \dots         &      0        &     0     \\
\vdots    & \vdots      & \vdots & \vdots        & \ddots        &   \vdots      &  \vdots   \\
0         &   0         &   0    & \lambda_{N-3} &      1        & \mu_{N-3}     &    0      \\
0         &   0         &   0    &   0           & \lambda_{N-2} &      1        & \mu_{N-2} \\
0         &   0         &   0    &   0           &   0           &  \frac{1}{2}  &  1     
\end{MyMatrix}
\begin{MyMatrix} y'_0 \\ y'_1 \\ y'_2 \\ \vdots \\ y'_{N-3} \\ y'_{N-2} \\ y'_{N-1} \end{MyMatrix} =
\begin{MyMatrix} d_0  \\  d_1 \\  d_2 \\ \vdots \\  d_{N-3} \\  d_{N-2} \\  d_{N-1} \end{MyMatrix} ,
\end{equation}
with
\begin{equation}
d_0 = \frac{3}{2} \frac{y_1-y_1}{h_0}, \ \ \ \ \ d_{N-1} = \frac{3}{2} \frac{y_{N-1}-y_{N-2}}{h_{N-1}}.
\end{equation}

\section{Bicubic Splines}
It is possible to extend the cubic spline interpolation method to
functions of two variables, i.e. $F(x,y)$.  In this case, we have a
rectangular mesh of points given by $F_{ij} \equiv F(x_i,y_j)$.  In
the case of 1D splines, we needed to store the value of the first
derivative of the function at each point, in addition to the value.
In the case of {\em bicubic splines}, we need to store four
quantities for each mesh point:  
\begin{eqnarray}
F_{ij}    & \equiv & F(x_i, y_i)            \\
F^x_{ij}  & \equiv & \partial_x F(x_i, y_i) \\
F^y_{ij}  & \equiv & \partial_y F(x_i, y_i) \\
F^{xy}    & \equiv & \partial_x \partial_y F(x_i, y_i)
\end{eqnarray}

Consider the point $(x,y)$ at which we wish to interpolate $F$.  We
locate the rectangle which contains this point, such that $x_i <= x <
x_{i+1}$ and $y_i <= x < y_{i+1}$.  Let 
\begin{eqnarray}
h & \equiv & x_{i+1}-x_i \\
l & \equiv & y_{i+1}-y_i \\
u & \equiv & \frac{x-x_i}{h} \\
v & \equiv & \frac{y-y_i}{l}
\end{eqnarray}
Then, we calculate the interpolated value as
\begin{equation}
F(x,y) = 
\begin{MyMatrix}
p_1(u) \\ p_2(u) \\ h q_1(u) \\ h q_2(u) 
\end{MyMatrix}^T
\begin{MyMatrix}
F_{i,j}     & F_{i+1,j}     & F^y_{i,j}      & F^y_{i,j+1}     \\
F_{i+1,j}   & F_{i+1,j+1}   & F^y_{i+1,j}    & F^y_{i+1,j+1}   \\
F^x_{i,j}   & F^x_{i,j+1}   & F^{xy}_{i,j}   & F^{xy}_{i,j+1}  \\
F^x_{i+1,j} & F^x_{i+1,j+1} & F^{xy}_{i+1,j} & F^{xy}_{i+1,j+1} 
\end{MyMatrix}
\begin{MyMatrix}
p_1(v)\\ p_2(v)\\ k q_1(v) \\ k q_2(v)
\end{MyMatrix}
\end{equation}
\subsection{Construction bicubic splines}
We now address the issue of how to compute the derivatives that are
needed for the interpolation.  The algorithm is quite simple.  For
every $x_i$, we perform the triadiagonal solution as we did in the 1D
splines to compute $F^y_{ij}$.  Similarly, we perform a tridiagonal
solve for every value of $F^x_{ij}$.  Finally, in order to compute the
cross-derivative we may {\em either} to the tridiagonal solve in the $y$
direction of $F^x_{ij}$, {\em or} solve in the $x$ direction for
$F^y_{ij}$ to obtain the cross-derivatives, $F^{xy}_{ij}$.  Hence,
only minor modifications to the $1D$ interpolations are necessary.

\section{Tricubic Splines}
Bicubic interpolation required two four-component vectors and a 4x4
matrix.  By extension, tricubic interpolation requires three
4-component vectors and a 4x4x4 tensor.  We summarize the forms of
these vectors below.
\begin{eqnarray}
h & \equiv & x_{i+1}-x_i \\
l & \equiv & y_{i+1}-y_i \\
m & \equiv & z_{i+1}-z_i \\
u & \equiv & \frac{x-x_i}{h} \\
v & \equiv & \frac{y-y_i}{l} \\
w & \equiv & \frac{z-z_i}{m}
\end{eqnarray}
\begin{eqnarray}
\vec{a} & = & 
\begin{MyMatrix}
p_1(u) & p_2(u) & h q_1(u) & h q_2(u) 
\end{MyMatrix}^T \\
\vec{b} & = & 
\begin{MyMatrix}
p_1(v) & p_2(v) & k q_1(v) & k q_2(v) 
\end{MyMatrix}^T \\
\vec{c} & = & 
\begin{MyMatrix}
p_1(w) & p_2(w) & l q_1(w) & l q_2(w) 
\end{MyMatrix}^T 
\end{eqnarray}
\begin{equation}
\begin{LMatrix}
A_{000} = F_{i,j,k}     & A_{001}=F_{i,j,k+1}     & A_{002}=F^z_{i,j,k}      & A_{003}=F^z_{i,j,k+1}      \\
A_{010} = F_{i,j+1,k}   & A_{011}=F_{i,j+1,k+1}   & A_{012}=F^z_{i,j+1,k}    & A_{013}=F^z_{i,j+1,k+1}    \\
A_{020} = F^y_{i,j,k}   & A_{021}=F^y_{i,j,k+1}   & A_{022}=F^{yz}_{i,j,k}   & A_{023}=F^{yz}_{i,j,k+1}   \\
A_{030} = F^y_{i,j+1,k} & A_{031}=F^y_{i,j+1,k+1} & A_{032}=F^{yz}_{i,j+1,k} & A_{033}=F^{yz}_{i,j+1,k+1} \\
                        &                         &                          &                            \\
A_{100} = F_{i+1,j,k}     & A_{101}=F_{i+1,j,k+1}     & A_{102}=F^z_{i+1,j,k}      & A_{103}=F^z_{i+1,j,k+1}      \\
A_{110} = F_{i+1,j+1,k}   & A_{111}=F_{i+1,j+1,k+1}   & A_{112}=F^z_{i+1,j+1,k}    & A_{113}=F^z_{i+1,j+1,k+1}    \\
A_{120} = F^y_{i+1,j,k}   & A_{121}=F^y_{i+1,j,k+1}   & A_{122}=F^{yz}_{i+1,j,k}   & A_{123}=F^{yz}_{i+1,j,k+1}   \\
A_{130} = F^y_{i+1,j+1,k} & A_{131}=F^y_{i+1,j+1,k+1} & A_{132}=F^{yz}_{i+1,j+1,k} & A_{133}=F^{yz}_{i+1,j+1,k+1} \\
                        &                         &                          &                            \\
A_{200} = F^x_{i,j,k}      & A_{201}=F^x_{i,j,k+1}      & A_{202}=F^{xz}_{i,j,k}      & A_{203}=F^{xz}_{i,j,k+1}    \\
A_{210} = F^x_{i,j+1,k}    & A_{211}=F^x_{i,j+1,k+1}    & A_{212}=F^{xz}_{i,j+1,k}    & A_{213}=F^{xz}_{i,j+1,k+1}  \\
A_{220} = F^{xy}_{i,j,k}   & A_{221}=F^{xy}_{i,j,k+1}   & A_{222}=F^{xyz}_{i,j,k}     & A_{223}=F^{xyz}_{i,j,k+1}   \\
A_{230} = F^{xy}_{i,j+1,k} & A_{231}=F^{xy}_{i,j+1,k+1} & A_{232}=F^{xyz}_{i,j+1,k}   & A_{233}=F^{xyz}_{i,j+1,k+1} \\
                        &                         &                          &                                      \\
A_{300} = F^x_{i+1,j,k}      & A_{301}=F^x_{i+1,j,k+1}      & A_{302}=F^{xz}_{i+1,j,k}    & A_{303}=F^{xz}_{i+1,j,k+1}   \\
A_{310} = F^x_{i+1,j+1,k}    & A_{311}=F^x_{i+1,j+1,k+1}    & A_{312}=F^{xz}_{i+1,j+1,k}  & A_{313}=F^{xz}_{i+1,j+1,k+1} \\
A_{320} = F^{xy}_{i+1,j,k}   & A_{321}=F^{xy}_{i+1,j,k+1}   & A_{322}=F^{xyz}_{i+1,j,k}   & A_{323}=F^{xyz}_{i+1,j,k+1}  \\
A_{330} = F^{xy}_{i+1,j+1,k} & A_{331}=F^{xy}_{i+1,j+1,k+1} & A_{332}=F^{xyz}_{i+1,j+1,k} & A_{333}=F^{xyz}_{i+1,j+1,k+1} 
\end{LMatrix}
\end{equation}
Now, we can write
\begin{equation}
F(x,y,z) = \sum_{i=0}^3 a_i \sum_{j=0}^3 b_j \sum_{k=0}^3 c_k \ A_{i,j,k} 
\end{equation}
The appropriate derivatives of $F$ may be computed by a generalization
of the method used for bicubic splines above.

\end{document}
