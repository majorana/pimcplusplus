\documentclass{article}
\usepackage{bbm}
\usepackage{amsmath}
\title{Plane wave band structure calculations with pseudohamiltonians}
\author{Kenneth P. Esler Jr.}

\begin{document}
\maketitle

\section{Introduction}
In order to perform use a fixed-phase nodal restriction for our PIMC
simulation, we need a method to compute reasonable electronic wave
functions, parameterized by the positions of the ions,
$\{\mathbf{R_i}\}$ and the ``twist vector'', $\mathbf{k}$.  By
``reasonable'', we mean to say that we wish the wave functions to have
the appropriate symmetries and capture the dominant effects of the ion
positions.  These goals can be achieved by working in a mean-field
approximation, in which the wave functions can be written as a
Slater-determinant of single-particle orbitals.
\begin{equation}
\Psi_k(\mathbf{R}) = \det \dots
\end{equation}
In order to determine the orbitals, $u_\mathbf{k}^n(\mathbf(r))$, we
expand the functions in a complete basis, which we truncate at an
appropriate size to retain both accuracy and computational tractability.

\subsection{Plane wave elements of the PH}
For decades, the plane-wave basis has been the physicist's tool of
choice for small to medium problems.  This basis fits very naturally
with the periodic simulation cell, is systematically improvable, and
can be utilized with great efficiency, thanks to the development of
the Fast Fourier Transform (FFT).  Plane-wave methods are extremely
mature and well-understood in the condensed matter community.

Foulkes and Schluter give the plane-wave coefficients for an electron
interacting with a pseudohamiltonian\cite{FS}:
\newcommand{\vk}{\mathbf{k}}
\newcommand{\vG}{\mathbf{G}}
\newcommand{\vGp}{\mathbf{G'}}
\newcommand{\vg}{\mathbf{g}}
\newcommand{\vr}{\mathbf{r}}
\begin{eqnarray}
V^\text{PH}_\text{single}(\vk; \vG, \vGp) & = &
\int e^{-i(\vk+\vG)\cdot \vr} \left[ -\frac{1}{2} \nabla a \nabla +
  \frac{1}{2r^2} b \hat{L}^2 + V \right] e^{i(\vk+\vGp)\cdot r} \, d^3
\vr \nonumber \\
& = & \frac{1}{2} (\vk + \vG)^T \cdot \mathbf{F}(\vG - \vGp)\cdot (\vk
+ \vGp) + V(|\vG - \vGp|),
\end{eqnarray}
where $\mathbf{F}$ is a $3\times 3$ tensor given by
\begin{equation}
\begin{split}
\mathbf{F}(\vG - \vGp) = & \left[ \rule{0cm}{0.33cm} a\left(|\vG-\vGp|\right) +
  b_\perp\left(|\vG-\vGp| \right) \right]\mathbbm{I}_3 \\
& + \vg \left[b_\parallel\left(|\vG-\vGp|\right) - 
  b_\perp\left( |\vG -\vGp|\right)   \right]\vg^T.
\end{split}
\end{equation}
\begin{eqnarray}
V(|\vG - \vGp|) & = & \int_0^\infty V(r) j_0(|\vG - \vGp|r) 4\pi r^2
\, dr \\
a(|\vG-\vGp|) & = &\int_0^{r_c} a(r) j_0(|\vG-\vGp|r) 4\pi r^2 \, dr \\
b_\perp(|\vG-\vGp|) & = &
\int_0^{r_c} b(r)\left[ {\textstyle \frac{2}{3}}
  j_0(|\vG-\vGp|r)-{\textstyle\frac{1}{3}}j_2(|\vG - \vGp|r)
  \right]4\pi r^2 \, dr\\
b_\parallel(|\vG-\vGp|) & = &
\int_0^{r_c} b(r)\left[ {\textstyle \frac{2}{3}}
  j_0(|\vG-\vGp|r)+{\textstyle\frac{2}{3}}j_2(|\vG - \vGp|r) \right]4\pi r^2 \, dr.
\end{eqnarray}
Here, $j_0$ and $j_2$ are spherical bessel functions, and $\vg = (\vG
- \vGp)/ |\vG -\vGp|$.  The integrals can be precomputed numerically
at the beginning of a run.

The above matrix elements is for an electron interactive with a single
ion located at the origin.  To account for several ions located as
positions $\{\mathbf{R}_m \}$, we can simply multiply the matrix
elements of the single-ion potential, $V^\text{PH}_\text{single}$ by
the {\em structure factor}, $S(\vG - \vGp)$.
\begin{equation}
V^\text{PH}_\text{multiple}(\vk;\vG,\vGp) = S(\vG-\vGp) V^\text{PH}_\text{single}
(\vk;\vG,\vGp)
\end{equation}
where
\begin{equation}
S(\vG - \vGp) = \sum_m e^{i(\vG -\vGp)\cdot R_m}.
\end{equation}

\section{The conjugate gradient method}
In most band-structure application, the number of bands (eigenvectors)
required is almost always much smaller than the number of basis
functions.  For this reason, direct matrix diagonalization is usually
not the most efficient means of solution.  More efficienct schemes are
available which find eigenvectors by performing an iterative search to
minimize the variational energy. In particular, Teter, Payne, and Allan
introduced a very efficient scheme based on the well-known {\em
  conjugate gradients} method\cite{TPA}.

A conjugate gradient approach has the additional advantage that the
Hamiltonian need not be explicitly stored in memory.  As the number of
plane waves gets large, storing $H$ could be a severe limitation.
Rather, we need only to be able to apply $H$ to a vector $c$.  As it
turns out, this operation can be done extremely efficiently using fast
Fourier transforms (FFTs).  While applying $H$ to $c$ would take
$\mathcal{O}(N^2)$ operators to do with convenction matrix-vector
multiplication, the operation can be done in $\mathcal{O}(N \ln N)$
operations using FFTs.
\subsection{Using FFTs}
We will begin with the potential part of the Hamiltonian.  The main
power of the method comes from the fact that while $\hat{V}$ is a
dense operator in reciporical space, it is diagonal in real space.
Hence, we may FFT our vector into real space, apply $V(r)$ by simple
multiplication, and then FFT the resultant vector back into
reciporical space.  If done properly, this give the numerically
identical result as doing the entire operation in reciporical space.

For local potentials, the kinentic energy operator is diagonal is
reciporical space, so that it's application requires only
$\mathcal{O}(N)$ operations.  Because PHs have position-dependent
masses, however, the pseudo-kinetic energy operator is not diagonal in
reciporical space.  With the appropriate use of FFTs, however, its
application can be done in $\mathcal{O}(N \ln N)$ operators, just as
the potential, but with a significantly larger prefactor.  Applying
the pseudo-kinetic operator, $\hat{K}^\text{PH} \equiv \frac{1}{2}
(\vk + \vG) \cdot F(\vG-\vG')\cdot (\vk+\vGp)$ to a vector
$\mathbf{c}$ can be accomplished through the following steps:
\begin{enumerate}
\item Compute the $F(\vG-\vGp)$ tensor in reciporical space. \label{computeF}
\item FFT the tensor into real space. \label{FFTF}
\item Multiply each element, $c_\mathbf{G}$ by $(\vk + \vGp)$, yielding
  a $3\times N$ component vector.
\item FFT to real space.  Call this $(\nabla c)(\vr)$.
\item Apply the $3\times 3$ tensor $F(\vr)$ to $(\nabla c)(\vr)$.
\item Inverse FFT back into reciporical space.  Call this
  $(\mathbf{F}c)_\vG$.
\item Finally dot with $\frac{1}{2}(\vk + \vG)$ to yield
  $(\hat{K}^\text{PH} \mathbf{c})_\vG$. 
\end{enumerate}
Note that steps \ref{computeF} and \ref{FFTF} need only done each time
the ion positions or $\vk$-point are changed.


\subsection{Parallelization}


\section{Results}

\begin{thebibliography}{9}
\bibitem{FS}  Foulkes, Schluter. {\em Phys. Rev. B.} {\bf 40}, 11505 (1990)
\bibitem{TPA} Teter, Payne, Allan. Phys. Rev. B {\bf 40}, 12255 (1989)

\end{thebibliography}
\end{document}
