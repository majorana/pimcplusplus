\documentclass{article}
\usepackage{amsmath}
\title{Notes on real-space projection for nonlocal pseudopotentials}
\author{Ken Esler}

\newcommand{\vG}{\mathbf{G}}
\newcommand{\vk}{\mathbf{k}}
\newcommand{\vl}{\mathbf{l}}
\newcommand{\vr}{\mathbf{r}}

\begin{document}
\maketitle
The nonlocal energy contribute from the $i^\text{th}$ atom from band
$n$ can be given by
\begin{equation}
\epsilon_{\mathbf{k},n}^i = \sum_l \sum_{m=-l}^{+l} E_l Z_{lm}^*
Z_{lm}, 
\end{equation}
where
\begin{equation}
Z_{lm} = \sum_\vG \lambda_{lm}(|\vG + \vk|) c_{\vG + \vk},
\end{equation}
with
\begin{equation}
\lambda_{lm}(\vG + \vk) = \frac{4\pi i^l}{\sqrt{\Omega_\text{cell}}}
\zeta(|\vG + \vk|) Y_{lm}(\theta_{\vG+\vk},\phi_{\vG+\vk}) e^{i(\vG +
  \vk)\cdot \mathbf{\tau}_i},
\end{equation}
and
\begin{equation}
  \zeta(q) = \int_0^\infty r^2\, \zeta(r) j_l(qr) \, dr,
\end{equation}
where $\mathbf{\tau}_i$ is the position of atom $i$.

Recall that 
\begin{equation}
\int_0^\infty q^2 j_l(qr) j_l(qr') dq = \frac{\pi}{2r^2} \delta(r-r').
\end{equation}
Then we may recover $\zeta_l(r)$ by writing
\begin{eqnarray}
\int_0^\infty q^2 \zeta_l(q) j_l(qr) dq & = & \int_0^\infty dq\, q^2
\int_0^\infty dr\, {r'}^2 \zeta_l(r') j_l(qr) j_l(qr') \\
& =  & \int_0^\infty dr'\, {r'}^2 \zeta_l(r') \int_0^\infty dq\, q^2
j_l(qr) j_l(qr')\\ 
& = & \int_0^\infty dr'\, {r'}^2 \zeta_l(r') \frac{\pi}{2r^2}
\delta(r-r') \\
& = & \frac{\pi}{2} \zeta_l(r).
\end{eqnarray}
Thus
\begin{equation}
\zeta_l(r) = \frac{2}{\pi} \int_0^\infty q^2 \zeta_l(q) j_l(qr) \, dr.
\end{equation}

We can then approximate $Z_{lm}$ in real space by writing
\begin{equation}
U_{lm} = \Omega_\text{mesh} \sum_\vl \zeta_l(|\vl - \mathbf{\tau}_i|) 
Y_{lm}(\dots) \Psi_{\vk,n}(\vl),
\end{equation}
where the $\vl$ are the FFT mesh points.  We may write
\begin{equation}
\Psi_{\vk,n}(\vl) = \sum_\vG c_{\vG+\vk,n} e^{i(\vG + \vk)\cdot \vl}
\end{equation}
and

\section{Filtering fourier components}
If the real-space projection is to match the reciprocal-space one,
$\zeta(q)$ must be zero for $q > G_\text{max FFT}$.  Otherwise, we
will have an FFT aliasing problem.  One approach would be to simply
transform to $\zeta(q)$, zero out $\zeta(q)$ for $q > G_\text{max
FFT}$, and then transform back to $\zeta(r)$.  This will indeed solve
the aliasing problem, but when we transform back to $\zeta(r)$, this
function will no longer go to zero for $r>r_c$.  The trick of
King-Smith et al\cite{KingSmith} is to recognize that $U_{lm}$ does
not depend on $\zeta_l(q)$ for $G_\text{max} < q <
\Gamma_1-G_\text{max}$, were $\Gamma_1$ is typically $4 G_\text{max}$.
This is because $\Psi_{\vk,n}(\vr)$ has no Fourier components for $q >
G_\text{max}$.  Until $q$ is large enough to wrap around into small
$q$ through FFT aliasing, there is no contribution.  For convience,
define $\gamma = \Gamma_1 - G_\text{max}$

Because of this ambiguity, we may define
\begin{equation}
\chi(q) = 
\begin{cases}
\zeta(q) & \text{if } q \le G_\text{max} \\
0        & \text{if } q \ge \gamma       \\
\text{arbitary} & \text{if } G_\text{max} < q \gamma.
\end{cases}
\end{equation}
We may choose the arbitary values such that $\xi_l(r)$ decays very
quickly to zero beyond a cutoff radius, $R_0$.  King-Smith et al. give
the optimal choice for $\chi_l(q)$ in this range.

To optimize $\chi_l(q)$, we must solve the integral equation,
\begin{equation}
\int_0^{G_\text{max}} A(q,q') \chi_l(q')\, dq' = \frac{\pi}{2} q^2
\chi_l(q) -\int_{G_\text{max}}^\gamma A(q,q')\chi(q')\, dq',
\end{equation}
where
\begin{equation}
A(q,q') = q^2 {q'}^2 \int_0^{R_0} j_l(q r) r^2 q_l(q'r) \, dr.
\end{equation}
We can discretize the integral, writing
\begin{equation}
\Delta \!\! \sum_{q'=0}^{G_\text{max}} A_{q,q'}\chi_{q'} - \frac{\pi
    q^2}{2} \chi_q = -\Delta\!\!\!\!\!\sum_{q'=G_\text{max}}^{\gamma}
    A_{q,q'} \chi(q'),
\end{equation}
where $\Delta$ is the discretization step in $q$.
Define 
\begin{equation}
X_q \equiv -\Delta \sum_{q'=0}^{G_\text{max}} A_{q,q'}\chi_{q'} + \frac{\pi
    q^2}{2} \chi_q
\end{equation}
Define the matrix, $M$, as
\begin{equation}
M_{q,q'} = \Delta A(q,q'),
\end{equation}
where $q$ and $q'$ are equally spaced points between $G_max$ and
$\gamma$.  

For $l=0$,
\begin{equation}
A(q,q') = -\frac{q q'\left[q \cos(q
    R_0) \sin(q' R_0) - q' \cos(q' R_0) \sin(q R_0)\right]}{q^2 - q'^2}.
\end{equation}
For $l=1$,
\begin{multline}
A(q,q') = \frac{q q'}{(q^2 - {q'}^2)} \left[ q \cos(q'
  R_0)\sin(q R_0) - q' \cos(q R_0)\sin(q' R_0)\rule{0mm}{4mm}\right] \\
- \frac{\sin(q R_0)\sin(q' R_0)}{R_0}
\end{multline}
For $l=2$,
\begin{multline}
A(q,q') = \frac{1}{q q'(q^2-q'^2) R_0^3} \left\{ \sin(q R_0) \left[q'
  R_0 \left(-3 q'^2 + q^2(3+q'^2 R_0^2)\right) - q R_0 \cos(q
  R_0)\left[3 q'(q^2-q'^2)R_0 \cos(q'R0) + 3q'^2 + q^2(-3 + q'^2
  R_0^2)\right]\sin(q' R_0)\right] \right\}
\end{multline}

\begin{thebibliography}{99}
  \bibitem{KingSmith} PRB {\bf 44}, 13063
\end{thebibliography}

\end{document}


