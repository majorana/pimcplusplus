\documentclass{article}
\title{Notes on the Birch fit}
\author{K.P. Esler}
\begin{document}
\maketitle
\section{Functional form}
Let $V_0$ be the equilibrium volume and $E_0$ the equilibrium static
energy.  We define an expansion in $t$, given by
\begin{equation}
t \equiv \left(\frac{V_0}{V}\right)^\frac{2}{3} - 1.
\end{equation}
The Birch energy function can then be given by
\begin{equation}
E(V) = E_0 + \sum_{n=0}^{N-1} b_n t^{n+2}.
\end{equation}
We have that
\begin{equation}
\frac{\partial t}{\partial V} = -\frac{2}{3V} 
\left(\frac{V_0}{V}\right)^{\frac{2}{3}} = -\frac{2}{3V}(t+1)
= -\frac{2}{3} V^{-\frac{5}{3}} V_0^{\frac{2}{3}}.
\end{equation}
The pressure can then be given by
\begin{eqnarray}
P(V) & = & -\frac{\partial t}{\partial V}\sum_{n=0}^{N-1} (n+2)b_n
t^{n+1} \\
& = & \frac{2}{3} V^{-\frac{5}{3}} V_0^{\frac{2}{3}} \sum_{n=0}^{N-1} (n+2)b_n
t^{n+1}.
\end{eqnarray}
The bulk modulus is given by
\begin{eqnarray}
K_T(V) & = & -V\left.\frac{\partial P}{\partial V}\right|_T \nonumber \\
& = & \frac{4}{9}V^\frac{-7}{3}V_0^{\frac{4}{3}}
\sum_{n=0}^{N-1} (n+1)(n+2)b_n t^n + \frac{10}{9}V^{-\frac{5}{3}} V_0^{\frac{2}{3}} \sum_{n=0}^{N-1} (n+2)b_n t^{n+1} \nonumber \\
& = & \frac{2}{9}V^{-\frac{5}{3}} V_0^{\frac{2}{3}}
\sum_{n=0}^{N-1} \left[2(n+1)(n+2)(t+1)t^n + 5(n+2)t^{n+1}\right]b_n \\
& = & \frac{2}{9V}(t+1) \sum_{n=0}^{N-1} (n+2)\left[2(n+1)t^n + 2(n+1)t^{n+1} + 5 t^{n+1}\right]b_n \nonumber \\
& = & \frac{2}{9V}(t+1) \sum_{n=0}^{N-1} (n+2)\left[2(n+1) + (2n+7)t\right]t^n b_n
\end{eqnarray}
%\begin{eqnarray}
%K_T(V) & = & -V\left.\frac{\partial P}{\partial V}\right|_T \nonumber \\
%& = & -V\left[\frac{2}{9}V^{-\frac{4}{3}}V_0^{-\frac{2}{3}} \sum
%\sum_{n=0}^{N-1} (n+2)b_n t^{n+1} -
%\frac{4}{9}V^{-\frac{2}{3}}V_0^{-\frac{4}{3}} \sum_{n=0}^{N-1}
%(n+2)(n+1) b_n t^n \right]\nonumber \\
%& = & -\frac{2}{9} V V^{-\frac{4}{3}} V_0^{-\frac{2}{3}} \left[
%\sum_{n=0}^{N-1} (n+2)b_n t^{n+1} - 2(t+1)\sum_{n=0}^{N-1} (n+2)(n+1)
%b_n t^n \right] \\
%& = & -\frac{2}{9} V^{-\frac{1}{3}} V_0^{-\frac{2}{3}} \left[
%\sum_{n=0}^{N-1} (n+2)b_n t^n \left[t -2(n+1)(t+1)\right]
%\right]
%\end{eqnarray}

\end{document}
