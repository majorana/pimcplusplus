\documentclass{article}
\title{Plane wave band structure calculations with pseudohamiltonians}
\author{Kenneth P. Esler Jr.}

\begin{document}
\maketitle

\section{Introduction}
In order to perform use a fixed-phase nodal restriction for our PIMC
simulation, we need a method to compute reasonable electronic wave
functions, parameterized by the positions of the ions,
$\{\mathbf{R_i}\}$ and the ``twist vector'', $\mathbf{k}$.  By
``reasonable'', we mean to say that we wish the wave functions to have
the appropriate symmetries and capture the dominant effects of the ion
positions.  These goals can be achieved by working in a mean-field
approximation, in which the wave functions can be written as a
Slater-determinant of single-particle orbitals.
\begin{equation}
\Psi_k(\mathbf{R}) = \det \dots
\end{equation}
In order to determine the orbitals, $u_\mathbf{k}^n(\mathbf(r))$, we
expand the functions in a complete basis, which we truncate at an
appropriate size to retain both accuracy and computational tractability.

For decades, the plane-wave basis has been the physicist's tool of
choice for small to medium problems.  This basis fits very naturally
with the periodic simulation cell, is systematically improvable, and
can be utilized with great efficiency, thanks to the development of
the Fast Fourier Transform (FFT).  Plane-wave methods are extremely
mature and well-understood in the condensed matter community.


\section{The conjugate gradient method}

\section{Using FFTs}

\section{Results}


\end{document}
