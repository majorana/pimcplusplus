\documentclass{article}
\usepackage{amsmath}
\title{Ewald Breakup for Long-Range Potentials in PIMC}
\author{Kenneth P. Esler Jr.}
\date{\today}
\begin{document}
\maketitle
Consider a group of particles interacting with long-ranged central
potentials, $v^{\alpha \beta}(|r^{\alpha}_i - r^{\beta}_j|)$, where the Greek superscripts
represent the particle species (eg. $\alpha=\text{electron}$,
$\beta=\text{proton}$), and Roman subscripts refer to particle number
within a species.  We can then write the total interaction energy for
the system as,
\newcommand{\vr}{\mathbf{r}}
\newcommand{\vk}{\mathbf{k}}
\begin{equation}
V = \sum_\alpha \left\{\sum_{i<j} v^{\alpha\alpha}(|\vr^\alpha_i - \vr^\alpha_j|) +
\sum_{\beta<\alpha} 
\sum_{i,j} v^{\alpha \beta}(|\vr^{\alpha}_i - \vr^{\beta}_j|) \right\}
\end{equation}
\newcommand{\va}{\mathbf{a}}
\newcommand{\vb}{\mathbf{b}}
\newcommand{\vL}{\mathbf{L}}
\subsection{The Long-Range Problem}
Consider such a system in periodic boundary conditions in a cell
defined by primitive lattice vectors $\va_1$, $\va_2$, and $\va_3$.
Let $\vL \equiv n_1 \va_1 + n_2 \va_2 + n_3\va_3$ be a direct lattice
vector.  Then the interaction energy per cell for the periodic system
is given by
\begin{equation}
\begin{split}
V = & \sum_\vL \sum_\alpha \left\{ 
\overbrace{\sum_{i<j} v^{\alpha\alpha}(|\vr^\alpha_i - \vr^\alpha_j + \vL|)}^{\text{homologous}} +
\overbrace{\sum_{\beta<\alpha} 
\sum_{i,j} v^{\alpha \beta}(|\vr^{\alpha}_i - \vr^{\beta}_j+\vL|)}^{\text{heterologous}}
\right\}  \\
& + \underbrace{\sum_{\vL \neq \mathbf{0}} \sum_\alpha N^\alpha v^{\alpha \alpha} (|\vL|)}_\text{Madelung}
\end{split}
\label{eq:direct},
\end{equation}
where $N^\alpha$ is the number particles of species $\alpha$.
If the potentials $v^{\alpha\beta}(r)$ are indeed long-range, the
summation over direct lattice vectors will not converge in this naive
form.  A solution to the problem was positted by Ewald.  We break the
central potentials into two pieces -- a short range and a long range
part define by
\begin{equation}
v^{\alpha \beta}(r) = v_s^{\alpha\beta}(r) + v_l^{\alpha \beta}(r).
\end{equation}
We will perform the summation over images for the short-range part in
real space, while performing the sum for the long-range part in
reciprocal space.  For simplicity, we choose $v^{\alpha \beta}_s(r)$
so that it is identically zero at the half the box length.  This
eliminates the need to sum over images in real space. In this letter,
we develop the details of the calculation and provide a way for
integrating this into a Path Integral Monte Carlo simulation.

\section{Reciprocal-Space Sums}
\subsection{Heterologous terms}
We begin with (\ref{eq:direct}), starting with the heterologous terms,
i.e. the terms involving particles of different species.  The
short-range terms are trivial, so we neglect them here.
\begin{equation}
\text{heterologous} = \frac{1}{2} \sum_{\alpha \neq \beta} \sum_{i,j} \sum_\vL
v^{\alpha\beta}_l(\vr_i^\alpha - \vr_j^\beta + \vL)
\end{equation}
We insert the resolution of unity in real space twice,
\begin{eqnarray}
\text{heterologous} & = & \frac{1}{2}\sum_{\alpha \neq \beta} \int_\text{cell} d\vr \, d\vr' \, \sum_{i,j}
\delta(\vr_i^\alpha - \vr) \delta(\vr_j^\beta-\vr') \sum_\vL
v^{\alpha\beta}_l(|\vr - \vr' + \vL|) \\
& = & \frac{1}{2\Omega^2}\sum_{\alpha \neq \beta} \int_\text{cell} d\vr \, d\vr' \, \sum_{\vk, \vk', i, j} e^{i\vk\cdot(\vr_i^\alpha
  - \vr)} e^{i\vk'\cdot(\vr_j^\beta - \vr')} \sum_\vL
v^{\alpha\beta}_l(|\vr - \vr' + \vL|) \nonumber \\
& = & \frac{1}{2\Omega^2} \sum_{\alpha \neq \beta} \int_\text{cell} d\vr \, d\vr'\,
\sum_{\vk, \vk', \vk'', i, j} e^{i\vk\cdot(\vr_i^\alpha - \vr)}
e^{i\vk'\cdot(\vr_j^\beta-\vr')} e^{i\vk''\cdot(\vr -\vr')}
v^{\alpha\beta}_{\vk''}, \nonumber.
\end{eqnarray}
Here, the $\vk$ summations are over reciprocal lattice vectors given
by $\vk = m_1 \vb_1 + m_2\vb_2 + m_3\vb_3$, where
\begin{eqnarray}
\vb_1 & = & 2\pi \frac{\va_2 \times \va_3}{\va_1 \cdot (\va_2 \times
  \va_3)} \nonumber \\
\vb_2 & = & 2\pi \frac{\va_3 \times \va_1}{\va_1 \cdot (\va_2 \times
  \va_3)} \\
\vb_3 & = & 2\pi \frac{\va_1 \times \va_2}{\va_1 \cdot (\va_2 \times
  \va_3)} \nonumber.
\end{eqnarray}
We note that $\vk \cdot \vL = 2\pi(n_1 m_1 + n_2 m_2 + n_3 m_3)$. 

\begin{eqnarray}
v_{k''}^{\alpha \beta} & = & 
\frac{1}{\Omega} \int_{\text{cell}} d\vr'' \sum_\vL
e^{-i\vk''\cdot(|\vr''+\vL|)} v^{\alpha\beta}(|\vr''+\vL|), \\
& = & \frac{1}{\Omega} \int_\text{all space} d\tilde{\vr} \, 
    e^{-i\vk'' \cdot \tilde{\vr}} v^{\alpha\beta}(\tilde{r}), \label{eq:vk}
\end{eqnarray}
where $\Omega$ is the volume of the cell. Here we have used the fact
that summing over all cells of the integral over the cell is
equivalent to integrating over all space.
\begin{equation}
\text{hetero} = \frac{1}{2\Omega^2} \sum_{\alpha \neq \beta}
\int_\text{cell} d\vr \, d\vr' \, \sum_{\vk, \vk', \vk'', i, j}
e^{i(\vk \cdot \vr_i^\alpha + \vk' \cdot\vr_j^\beta)} e^{i(\vk''-\vk)\cdot \vr}
e^{-i(\vk'' + \vk')\cdot \vr'} v^{\alpha \beta}_{\vk''}.
\end{equation}
We have
\begin{equation}
\frac{1}{\Omega} \int d\vr \  e^{i(\vk -\vk')\cdot \vr} =
\delta_{\vk,\vk'},
\end{equation}
Then, performing the integrations we have
\begin{eqnarray}
\text{hetero} = \frac{1}{2} \sum_{\alpha \neq \beta}
\sum_{\vk, \vk', \vk'', i, j}
e^{i(\vk \cdot \vr_i^\alpha + \vk' \cdot\vr_j^\beta)} \delta_{\vk,\vk''}
\delta_{-\vk', \vk''} v^{\alpha \beta}_{\vk''}.
\end{eqnarray}
We now separate the summations, yielding
\begin{equation}
\text{hetero} = \frac{1}{2} \sum_{\alpha \neq \beta} \sum_{\vk, \vk'}
\underbrace{\left[\sum_i e^{i\vk  \cdot \vr_i^\alpha} \rule{0cm}{0.705cm}
    \right]}_{\rho_\vk^\alpha}
\underbrace{\left[\sum_j e^{i\vk' \cdot \vr_j^\beta} \right]}_{\rho_{\vk'}^\beta}
 \delta_{\vk,\vk''} \delta_{-\vk', \vk''} v^{\alpha
  \beta}_{\vk''}.
\end{equation}
Summing over $\vk$ and $\vk'$, we have
\begin{equation}
\text{hetero} = \frac{1}{2} \sum_{\alpha \neq \beta} \sum_{\vk''}
\rho_{\vk''}^\alpha \, \rho_{-\vk''}^\beta v_{k''}^{\alpha \beta}.
\end{equation}
We can simplify the calculation a bit further by rearranging the
sums over species,
\begin{eqnarray}
\text{hetero} & = & \frac{1}{2} \sum_{\alpha > \beta} \sum_{\vk}
\left(\rho^\alpha_\vk \rho^\beta_{-\vk} + \rho^\alpha_{-\vk}
\rho^\beta_\vk\right) v_{k}^{\alpha\beta} \\
& = & \sum_{\alpha > \beta} \sum_\vk \mathcal{R}e\left(\rho_\vk^\alpha
\rho_{-\vk}^\beta\right)v_k^{\alpha\beta} .
\end{eqnarray}


\subsection{Homologous Terms}
We now consider the terms involving particles of the same species
interacting with each other.  The algebra is very similar to that
above, with the slight difficulty of avoiding the self-interaction term.
\begin{eqnarray}
\text{homologous} & = & \sum_\alpha \sum_L \sum_{i<j} v_l^{\alpha
  \alpha}(|\vr_i^\alpha - \vr_j^\alpha + \vL|) \\
 & = & \frac{1}{2} \sum_\alpha \sum_L \sum_{i\neq j} v_l^{\alpha
  \alpha}(|\vr_i^\alpha - \vr_j^\alpha + \vL|) 
\end{eqnarray}
\begin{eqnarray}
\text{homologous} & = & \frac{1}{2} \sum_\alpha \sum_L 
\left[
-N^\alpha v_l^{\alpha \alpha}(|\vL|)  + \sum_{i,j} v^{\alpha \alpha}_l(|\vr_i^\alpha - \vr_j^\alpha + \vL|)
  \right] \\
& = & \frac{1}{2} \sum_\alpha \sum_\vk \left(|\rho_k^\alpha|^2 - N
\right) v_k^{\alpha \alpha}
\end{eqnarray}

\subsection{Madelung Terms}
Let us now consider the Madelung term for a single particle of species
$\alpha$.  This term corresponds to the interaction of a particle with
all of its periodic images.  
\begin{eqnarray}
v_M^{\alpha} & = & \frac{1}{2} \sum_{\vL \neq \mathbf{0}} v^{\alpha
  \alpha}(|\vL|) \\
& = & \frac{1}{2} \left[ -v_l^{\alpha \alpha}(0) + \sum_\vL v^{\alpha
  \alpha}(|\vL|) \right] \\
& = & \frac{1}{2} \left[ -v_l^{\alpha \alpha}(0) + \sum_\vk v^{\alpha
  \alpha}_\vk \right]  
\end{eqnarray}

\section{Combining Terms}
Here, we sum all of the terms we computed in the sections above,
\begin{eqnarray}
V & = & \sum_{\alpha > \beta} \left[\sum_{i,j} v_s(|\vr_i^\alpha
  -\vr_j^\beta|) + \sum_\vk \mathcal{R}e\left(\rho_\vk^\alpha
  \rho_{-\vk}^\beta\right)v^{\alpha\beta}_k    \right] \nonumber \\
& + & \sum_\alpha \left[ N^\alpha v_M^\alpha + \sum_{i>j} v_s(|\vr_i^\alpha -
  \vr_j^\alpha|) + \sum_\vk \left( |\rho_\vk^\alpha|^2 -
  N\right) v^{\alpha\alpha}_\vk \right] \nonumber \\
& = & \sum_{\alpha > \beta} \left[\sum_{i,j} v_s(|\vr_i^\alpha
  -\vr_j^\beta|) + \sum_\vk \mathcal{R}e\left(\rho_\vk^\alpha
  \rho_{-\vk}^\beta\right) v^{\alpha \beta}_k   \right] \nonumber \\
& + & \sum_\alpha \left[ -\frac{N^\alpha v_l^{\alpha \alpha}(0)}{2}  + \sum_{i>j} v_s(|\vr_i^\alpha -
  \vr_j^\alpha|) + \frac{1}{2} \sum_\vk |\rho_\vk^\alpha|^2 v^{\alpha\alpha}_\vk \right] 
\end{eqnarray}

\section {Computing the Reciprocal Potential}
Now we return to (\ref{eq:vk}).  Without loss of generality, we define
for convenience $\vk = k\hat{\mathbf{z}}$.
\begin{equation}
v^{\alpha \beta}_k = \frac{2\pi}{\Omega} \int_0^\infty dr \int_{-1}^1
  d\cos(\theta) \ r^2 e^{-i k r \cos(\theta)} v_l^{\alpha \beta}(r)
\end{equation}
We do the angular integral first.  By inversion symmetry, the
imaginary part of the integral vanishes, yielding
\begin{equation}
v^{\alpha \beta}_k = \frac{4\pi}{\Omega k}\int _0^\infty dr\ r \sin(kr)
v^{\alpha \beta}_l(r).
\label{eq:vkint}
\end{equation}

\section{The Coulomb Potential}
For the case of the Coulomb potential, the above integral is not
formally convergent if we do the integral naively. We may remedy the
situation by including a convergence factor, $e^{-k_0 r}$.  For a
potential of the form $v^\text{coul}(r) = q_1 q_2/r$, this yields
\begin{eqnarray}
v^{\text{screened coul}}_k & = & \frac{4\pi q_1 q_2}{\Omega k} \int_0^\infty dr\ \sin(kr)
e^{-k_0r} \\ 
& = & \frac{4\pi q_1 q_2}{\Omega (k^2 + k_0^2)}
\end{eqnarray}
Allowing the convergence factor to tend to zero, we have
\begin{equation}
v_k^\text{coul} = \frac{4 \pi q_1 q_2}{\Omega k^2}
\end{equation}

For more generalized potentials with a coulomb tail, we cannot
evaluate (\ref{eq:vkint}) numerically but must handle the coulomb part
analytically.  In this case, we have
\begin{equation}
v_k^{\alpha \beta} = \frac{4\pi}{\Omega} 
\left\{ \frac{q_1 q_2}{k^2} + \int_0^\infty dr \ r \sin(kr) \left[ v_l^{\alpha \beta}(r) -
  \frac{q_1 q_2}{r} \right] \right\}
\end{equation}

\section{Adapting to PIMC}
\subsection{Pair actions}
\subsection{RPA improvements}
\section{Breakup Methods}
\subsection{Gaussian charge screening}
This original approach to the short and long-ranged breakup adds an
opposite screening charge of gaussian shape around each point charge.
It then removes the charge in the long-ranged part of the potential.
In this potential,
\begin{equation}
v_{\text{long}}(r) = \frac{Z_1 Z_2}{r} \text{erf}(\alpha r),
\end{equation}
where $\alpha$ is an adjustable parameter used to control how
short-ranged the potential should be.  If the box size is $L$, a
typical value for $\alpha$ might be $7/(LZ_1 Z_2)$. We should note
that this form for the long-ranged potential should also work for any
general potential with a coulomb tail, e.g. pseudo-Hamiltonian
potentials.  For this form of the long-ranged potential, we have in $k$-space
\begin{equation}
v_k = \frac{4\pi\exp\left[\frac{-k^2}{4\alpha^2}\right]}{\Omega k^2}.
\end{equation}

\subsection{Optimizing the breakup}



\end{document}
