\documentclass{article}
\usepackage{amsmath}
\usepackage{algorithm}
\usepackage{algorithmic}
\title{Correlated sampling for action differences with PIMC}
\author{K.P. Esler and B.K. Clark}

\newcommand{\vR}{\mathbf{R}}
\newcommand{\vr}{\mathbf{r}}

\begin{document}
\maketitle
\section{Motivation}
We wish to be able to ion configurations by averaging over electron
paths with the penalty method.  In order to do this, we need to be
able to estimate the action difference between two configurations of
the ions, $\vR_A$ and $\vR_B$, with the electronic degrees of freedom
integrated out.  In order for this method to be effe

\section{Sampling probability}
It is, in general, impossible to select the optimal importance
sampling function without an {\em a priori} knowledge of the
respective partition functions for the electronic system,
$Z_\text{elec}(\mathbf{R})$ as a function of the ion positions.
However, we can select a nearly optimal importance function that has
the desired properties.  Let $S_A$ and $S_B$ represent the imaginary
time action for the systems as with respective ion postions $\vR_A$
and $\vR_B$.
\begin{equation}
p^* = e^{-S_A} + e^{-S_B}
\end{equation}

\subsection{Time slice parallelization}
Let us now consider the quantity we will use in determining whether to
accept or reject a change in the electron paths,
\begin{equation}
\frac{p^*_\text{new}}{p^*_\text{old}} = 
\frac{e^{S_A^\text{new}} + e^{S_B^\text{new}}}{e^{S_A^\text{old}} +
  e^{S_B^\text{old}}}.
\end{equation}

Now, consider splitting our paths over several processors, such that
each processor holds several time slices.  In a usual PIMC simulation,
most moves may be made independently on each processor, since this
acceptance ratio may be factorized as a product of ratios calculated
from the portion of the path on each processor.  Since we now have a
sum of exponentials, we cannot make this factorization, and hence now
move can be exclusively ``local'' to a single processor.  We can,
however, reduce the communication overhead that would be incurred by
doing a naive global move in which we have to perform a collective sum
over all processors at every stage of bisection.

In order to see how, let us then define new quantities,
\begin{eqnarray}
\bar{S} & \equiv & \frac{S_A + S_B}{2} \\
\Delta S & \equiv & \frac{S_A - S_B}{2} 
\end{eqnarray}
Then, we may rewrite
\begin{equation}
p^* = e^{-\bar{S}}\left(e^{-\Delta S} + e^{+\Delta S} \right)
\end{equation}
Consider now that the action $S_A$ may be given by summing over the
actions, $S_A^i$, from processors 1 through $N$.  
\begin{equation}
S_A = \sum_{i=1}^N S_A^i,
\end{equation}
and similarly for $S_B$.  Then, our probability ratio may be written
as
\begin{equation}
\frac{p_*^\text{new}}{p_*^\text{old}} = 
\frac{\exp \left[-\sum_{i=1}^{N} \bar{S}^\text{new}_i \right]}
{\exp \left[-\sum_{i=1}^{N} \bar{S}^\text{old}_i \right]}
\left[
\frac{\exp \left(-\sum_{i=1}^N \Delta S_i^\text{new} \right) +
\exp \left(+\sum_{i=1}^N \Delta S_i^\text{new} \right)}
{\exp \left(-\sum_{i=1}^N \Delta S_i^\text{old} \right)+
\exp \left(+\sum_{i=1}^N \Delta S_i^\text{old} \right)} \right]
\end{equation}
We first note that we have factored the acceptance ratio into two
pieces, one involving the $\bar{S}$'s and one involving the $\Delta
S$'s. This factorization implies we can separate those pieces into two
acceptance stages -- first accept or reject based on the $\bar{S}$'s.
If we accept, we go on to accept or reject based on the $\Delta S$'s.
Furthermore, we note that the $\bar{S}$ stage may be further
factorized by processor:
\begin{equation}
\frac{\exp \left[-\sum_{i=1}^{N} \bar{S}^\text{new}_i \right]}
{\exp \left[-\sum_{i=1}^{N} \bar{S}^\text{old}_i \right]} =
\prod_{i=1}^N \exp \left[ \bar{S}_i^\text{old} - \bar{S}_i^\text{new}\right]
\end{equation}
This factorization implies that we may make the decision about whether
to accept the partial move on each processor independently.

Considered in the context of the multistage bisection move, each
processor begins at the highest bisection level.  At each level, each
processor may continue to the next bisection level, or ``opt out'' of
the collective move.  Once we get to the final bisection stage, we
must do a collective communication, summing the action changes on all
the processors that have made it to this stage.  Finally, we accept or
reject the move globally for all the processors simultaneously.  This
algorithm can be seen schematically in ...

\begin{algorithm}
\caption{Modified correlated bisection algorithm}
\begin{algorithmic}
\STATE level $\leftarrow$ highest level
\WHILE{(NOT rejected) AND (bisection level $>0$)}
  \STATE Construct new path at this level
  \STATE Calculate new and old level actions given by $\bar{S}_i$, 
  where $i$ is my processor number.  Do not include nodal action in
  calculation.
  \STATE Accept or reject local level move.
  \STATE level $\leftarrow$ level-1
\ENDWHILE

\IF{ accepted at level 1 }
  \STATE Send my change in $\Delta S$ to collective sum.
  \STATE Accept or reject whole move globally based on collective sums.
\ELSE
  \STATE Send 0 to collective sum for change in $\Delta S$.
\ENDIF

\end{algorithmic}
\end{algorithm}

This scheme has the advantages that it avoid collective communication
at each stage.  Futhermore, it prevents very unfavorable bisections
(such as those in which electrons overlap) from causing a global
rejection of the move.




\section{Twist averaging}




\end{document}
