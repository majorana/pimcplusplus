\documentclass{book}
%\usepackage{verbatim}
%\usepackage{moreverb}
%\usepackage{boxedminipage}
%\usepackage{shadow}
\usepackage{fancybox}
\usepackage{float}
\usepackage{amsmath}
\usepackage{hyperref}


\renewcommand{\shadowsize}{2pt}
\author{Kenneth P. Esler, Jr. and Bryan K. Clark}
\title{pimc++}
%\subtitle{Path Integral Monte Carlo Simulation Tool}

\begin{document}
\maketitle
\chapter{Introduction}
{\em pimc++} is a general, high-performance tool to perform
fully-correlated simulations of quantum systems at finite temperature.
In essense, it samples the many-body thermal density matrix to compute
thermal equilibrium averages of quantum observables.  Written in very
object-oriented C++, it is intended to be a nearly universal tool that
can be easily extended to perform novel simulations.  Its current
features include:
\begin{itemize}
  \item Arbitrary number of particles, species, and timeslices,
        limited only by memory requirements.
  \item Time-slices can be distributed across processors using MPI to
        communicate, making possible efficient simulations with
        thousands of time-slices.
  \item Multiple ``clones'' of the same simulation may be run with
        automatic generation of different random number seeds.  This
        can be used to improve statistics.
  \item Very high-accuracy pair-actions can be used.  These are
        generated with the coupled squarer++/fitter++ package.
  \item Bose, Fermi, and Boltzmann statistics are supported.
  \item Output is efficiently and accurately stored in HDF5 format.
  \item Extremely flexible input file format allows specification of
        the Monte Carlo algorithm in the input.
  \item Free and periodic boundary conditions are specified
        independently for each dimension allowing slab and column
        geometries. 
  \item Statistical analysis and HTML report generation with
        automatic creation of embedded plots.
  \item 3D OpenGL-based visualization tool can be used for debugging
        and pedagogy.  Supports exporting to the POVray ray-tracer for 
        photo-realistic rendering.  Also supports MPEG4 movie generation.
\end{itemize}

\chapter{Installing}
\section{Dependencies}
In the spirit of the open source movement, {\em pimc++} makes use of a
number of freely-available libraries.  These libraries perform their
individual tasks very well and avoid the reinvention of the wheel.
The following libriaries need to be installed before attempting to
install {\em pimc++}:
\begin{itemize}
  \item {\bf BLAS/LAPACK:}  These are linear algebra routines for
    matrix multiplies, factorization, inversion, etc.  They are utilized
    in the plane-wave portions of the code, long-range optimized
    breakups, and fermion determinants.  \\
    (\href{http://www.netlib.org/lapack}{http://www.netlib.org/lapack})
  \item {\bf Gnu Scientific Library (GSL):}  Though this library is
    extensive, we use it mainly for computing special functions.\\
    (\href{http://http://www.gnu.org/software/gsl/}{http://http://www.gnu.org/software/gsl/})
  \item {\bf SPRNG:}  The Simple Parallel Random Number Generator is
    used to generate pseudo-random numbers for driving the Monte Carlo.
    It has the somewhat unique feature of being able to generate streams
    of random numbers that are shared between processors.\\
   (\href{http://esler.physics.uiuc.edu/downloads/sprng-2.0.tar.gz}{http://esler.physics.uiuc.edu/downloads/sprng-2.0.tar.gz})
  \item {\bf HDF5:}  The Hierarchical Data Format is used to store
    data in a platform-independent binary form.  \\
    (\href{http://hdf.ncsa.uiuc.edu/HDF5/}{http://hdf.ncsa.uiuc.edu/HDF5/})
  \item {\bf MPI:} ({\em Optional})  The Message Passing Interface is
    used for communication between nodes when doing parallel runs.
\end{itemize}

\section{Installing the Common library}
Many of the routines used in {\em pimc++} are shared with a number of
other codes in the suite, such as {\em squarer++}, {\em phgen++}, and
{\em pathvis++}.  Because of this, we packaged these routines in a
library we call {\em common} for lack of a better name.  The {\em common}
library must be installed before {\em pimc++}.  

The {\em common} library, {\em pimc++}, and all the other programs in
the suite use the standard autoconf/automake approach to building.
The basic setup involves doing\\
\rule{0.6cm}{0cm}\shadowbox{\begin{minipage}{4.4in}
\begin{verbatim}
./configure
make
make install
\end{verbatim}
\end{minipage}}\\
The \texttt{configure} script has numerous options.  They may be
listed by doing ``\texttt{./configure --help}''.

\chapter{Input Files}
\section{Introduction}
All of the file input and output in pimc++ is done through our 
C++ Input And Output ({\em CIAO}) library.  This library is designed
to provide a common programming interface to several hierachical data
formats.  In particular, we currently implement a custom ASCII format,
HDF5, and XML.  For input files, we use the ASCII format almost
exclusively.  For files which store heavy numeric data, such as the
pair action spline tables, we use HDF5.  XML is appealing for some
compatibility reasons, but we do not currently use it for anything.
The following sections will explain the use of our ASCII format.  

\section{Defining variables}
Upon looking at a sample input file, the reader will note a strong
resemblence to C++.  This was an intentional choice on the part of the
developers, since those who were familiar enough with C++ to extend
the code should feel quite at-home with this format.  There are very
few constructs to learn in the input files, yet we have found them to
be very versatile.  

Before diving into the details of the file format, let us begin with
an example.  In Figure~\ref{fig:InputEx}, we have a fictitious input
file in the {\em CIAO} ASCII format.
\begin{figure}
\shadowbox{\begin{minipage}{5in}
\begin{verbatim}
Section(System) {
  int NumParticles = 6;
  Array<double,1> Masses(6) = [ 0.5, 0.5, 0.5, 1.0, 1.0, 1.0 ];
\}

Section(Potentials) {
  Section(Potential) {
    string Type = "Coulomb";
    double q1 = 1.0;
    double q2 = -1.0;
  \}

  Section(Potential) {
    string Type = "Lennard-Jones";
    double sigma = 5.0;    // this is in units of bohr radii
    double epsilon = 3.2;  // and this is in Hartrees
  \}
\}
\end{verbatim}
\end{minipage}}
\caption{Example {\em CIAO} input file.}
\label{fig:InputEx}
\end{figure}

The most basic construct is a named variable to which a value is
assigned.  An example is \\
\rule{0.6cm}{0cm}\shadowbox{\parbox{4.4in}{\texttt{int NumParticles = 6;}}} \\
In this case, \texttt{NumParticles} is an {\em atomic} variable.
Currently, {\em CIAO} supports the atomic types {\texttt int, double,
  bool and string}.  It also supports array variables.  We have taken
the syntax for our array variables from the popular {\em blitz++}
template array package. \\
\rule{0.6cm}{0cm}\shadowbox{\parbox{4.4in}{\texttt{Array<double,1>
      Masses(6) = [0.5, 0.5, 0.5, 1.0, 1.0, 1.0]; }}} \\
Here, the \texttt{Array<double,1>} indicates a one-dimensional array
of doubles.  {\em CIAO} supports up to four-dimensional arrays.  The
\texttt{(6)} after the variable name, \texttt{Masses}, gives the size
of the array.  For a multidimensional array, the dimensions are
separated by commas, such as\\
\rule{0.6cm}{0cm}\shadowbox{\begin{minipage}{4.4in}
\begin{verbatim}
Array<double,2> Ident(3,3) = [ 1.0, 0.0, 0.0,
                               0.0, 1.0, 0.0,
                               0.0, 0.0, 1.0 ];
\end{verbatim}
\end{minipage}}\\
The blank lines are not required -- they are included only for
clarity.  

\section{Creating hierachy with sections}
Following the object-oriented paradigm, we find it to be extremely
useful to allow hierarchy in our input files.  We do this through the
use of {\em sections}.  A section is defined in the following way:\\
\rule{0.6cm}{0cm}\shadowbox{\begin{minipage}{4.4in}
\begin{verbatim}
  Section (Cattle) {
    Section (Cows) {
      Array<string,1> Names(2) = ["Bessie", "Bovina"];
      Array<int,1> NumSpots(2) = [1, 47];
    }
    Section (Bulls) {
      Array<string,1> Names(2) = ["Charger", "Red"];
      Array<int,1> NumSpots(2) = [13, 5];
    }
  }
\end{verbatim}
\end{minipage}}

\section{Format}
\section{Global parameters}
\section{The ``System'' section}
   The section descibes the system to be studied.  It specifies the
   size of the paths, the species and numbers of particles, the size
   of the box, etc.  Here, we will describe the required parameters
\begin{itemize}
   \item \texttt{NumTimeSlices}:  Integer variable giving the total
   number of time slices to use.  Thus
   $\texttt{tau}*\texttt{NumTimeSlices}$ gives the $\beta$ for the
   system.
   \item \texttt{Array<bool,1> IsPeriodic(NDIM)}:  This is an array of
   booleans of length 2 or 3, dependending on the dimensionality of
   the system, of whether the system is periodic in each of the
   cardinal directions, $\hat{x},\hat{y},\text{and } \hat{z}$
   \item \texttt{Section (Particles)}  This section describes all the
   particles in the system.  It is comprised of one or more
   \texttt{Species} sections.
     \subitem \texttt{Section (Species)}  Describes a group of
        identicle particles.  Examples include electrons, protons,
        or helium atoms.
        \subsubitem acdkajdf
\end{itemize}


\section{The ``Action'' section}
\subsection{Pair Actions}
You must include a pair action file for each pair of particle types.
If you have only one species in your system, (e.g. $^4He$ atoms), you
need only one pair action file.  If you have two different particle types
(e.g. electrons and protons), you will need three files
(e.g. electron-electron, electron-proton, and proton-proton).  These
files are specified by the variable \texttt{Array<string,1>
  PairActionFiles($n$)}, where $n$ is the number of files.  For
example, \\ 
\rule{0.6cm}{0cm}\shadowbox{\begin{minipage}{4.4in}
\begin{verbatim}
Array<string,1> PairActionsFiles(3) = 
    [ "e-e.h5", "e-p.h5", "p-p.h5" ];
\end{verbatim}
\end{minipage}}\\
These files are the ones generated by {\em fitter++}, which is
described in an accompanying document.

\subsection{Pair Actions (from David's squarer)}
PIMC++ is capable of reading Pair Actions from David's squarer output.
For each dm file used, a separate .PairAction file must be created.
This PairAction file must have the following structure:
\rule{0.0cm}{0.75cm}\rule{0.6cm}{0cm}\shadowbox{\begin{minipage}{4.4in}
\begin{verbatim}
Section (Fits)
{
  string Type = "DavidFit";
  string Daviddmfile="He4.95.dm";
  int NumOffDiagonalTerms=2 
  Section (Particle1)
  {
    string Name=''He3''
    double lambda = 6.059615;
    int Ndim = 3;
  }
 
  Section (Particle2)
  {
    string Name = "He";
    double lambda = 0.0;
    int Ndim = 3;
  }
 

}
\end{verbatim}
\end{minipage}}
The Type variable being set to DavidFit indicates that David's squarer
is to be used for input.  The variable Daviddmfile gives the name of
the density matrix file to use.  This file must be in the directory
from which youa re running the code. If it is not, a cryptic error
message such as assert(PAIO.ReadFile==false).  The variable
NumOffDiagonalTerms indicates how many off diagonal terms to use from
the dm file.  In the two particle sections it is important to use
Names that are consistent with the rest of the code. Currently, the
other variables must be included but are not actually used for any
purpose (as far as I am aware..they really should do a sanity check
against the code to make sure that what you are reading in is
consistent with other input).

\subsection{MaxLevels}
This integer variable gives the maximum number of levels which are
used in bisection-type moves.  It is required since we need to know
how may different $\tau$'s we need to read in from each pair action
file.  E.g.\\
\rule{0.6cm}{0cm}\shadowbox{\begin{minipage}{4.4in}
\begin{verbatim}
int MaxLevels = 6;
\end{verbatim}
\end{minipage}}
\subsection{Long-range variables}
For systems containing particles with long-range (e.g. coulomb)
interactions, there are several variables which are required:
\subsubsection{UseRPA:}
   This boolean variable indicates whether or not a Random Phase
 Approximation correction to the long-range part of the interaction
 should be applied.  It has not been extensively tested.\\
\rule{0.6cm}{0cm}\shadowbox{\begin{minipage}{4.4in}
\begin{verbatim}
bool UseRPA = false;
\end{verbatim}
\end{minipage}}\\
\subsubsection{NumBreakupKnots:}
This integer variable specifies how may spline knots should be used in
representing the optimal long-range action in real space.  A value of
about 10-20 is usually sufficient.  Too large a value could
potentially lead to numerical instabilities.\\
\rule{0.0cm}{0.75cm}\rule{0.6cm}{0cm}\shadowbox{\begin{minipage}{4.4in}
\begin{verbatim}
int NumBreakupKnots = 20;
\end{verbatim}
\end{minipage}}\\



\subsection{Nodal actions}
For systems involving fermions, we utilize implement the Restricted
Path Integral Monte Carlo method, RPIMC.  In this method, we force the
density matrix to have the same nodes (zero crossings) as a trial
density matrix.  In the case of complex trial functions, this is
generalized to enforcing that the phase of the sampled density matrix
matches some trial density matrix.  These methods are often referred
to as {\em fixed-node} and {\em fixed-phase}, since they were derived
from those ground-state methods.  

Since there are several different types of trial functions which can
be use for nodes/phases, each nodal action is specified in its own
section.  
\subsubsection{Free particle:}
The density matrix for a system of non-interacting fermions can be
computed exactly in free or periodic boundary conditions.  For
high-temperature or weakly interacting particles (such as in some
metals), the nodes from such a density matrix are fairly accurate.  To
specify these free-particle nodes, we need only two variables: \\ \\
\rule{0.6cm}{0cm}\shadowbox{\begin{minipage}{4.4in}
\begin{verbatim}
Section (NodalAction)
{
  string Type = "Free";
  string Species= "eup";
}
\end{verbatim}
\end{minipage}}
The first variable, \texttt{Type}, specifies we are using
free-particle nodes.  The second specifies which species is associated
with the nodal action.  If you have separate up and down electron
species (as you probably should), you need a
\texttt{Section(NodalAction)} for each species.

\subsection{Ground state:}
\rule{0.0cm}{0.75cm}\rule{0.6cm}{0cm}\shadowbox{\begin{minipage}{4.4in}
\begin{verbatim}
Section (NodalAction)
{
  string Type = "GROUNDSTATE";
  string IonSpecies  = "Na";
  string UpSpecies   = "eup";
  string DownSpecies = "edown";
  double kCut = 4.0;
}
\end{verbatim}
\end{minipage}}

In the limit of zero temperature, the density matrix can be written in
terms of the wave function for the system.  At low, but non-zero,
temperature, the using nodes from the ground state wave function is
often a very good approximation.  Above, we show how ground state
nodes are specified in the \texttt{Action} section.  First, the
\texttt{``Type''} is given as \texttt{``GroundState''}.  We next
specify the species of the ions.  The code will solve for the bands of
non-interacting electrons in the potential of the ions.  This
potential is determined from the pair action file.  Next, we specify
the {\em up} and {\em down} electron species.  Finally, we specify the
cutoff in $k$-space which we use for the band structure calculation.

\subsubsection{Fixed-phase:}
\rule{0.0cm}{0.75cm}\rule{0.6cm}{0cm}\shadowbox{\begin{minipage}{4.4in}
\begin{verbatim}
Section (NodalAction)
{
  string Type = "FIXEDPHASE";
  string IonSpecies  = "Na";
  string UpSpecies   = "eup";
  string DownSpecies = "edown";
  double kCut = 4.0;
  Array<double,1> kVec(3) = [ 0.04, 0.04, 0.04 ];
}
\end{verbatim}
\end{minipage}}\\
In the above example, we first specify the type as fixed-phase with
the \texttt{Type} string variable.  Next, we specify the names of the
species corresponding to up electrons, down electrons, and nuclei.
This well be used to perform a band structure calculation to construct
a trial wave-function.  Since we use a trial wave function instead of
the trial density matrix, the phases are independent of $\beta$, and
are thus probably only suitable for low-temperature calculations.

\subsection{UseBackground}
This boolean variable specifies whether or not we will include and
neutralizing charge background.  This should only be used for systems
which are not charge-neutral, such as jellium.

\section{The ``Moves'' section}

PIMC++ deals with moves and observables in a two steps. First, in this
section of the input file, the possible moves (observables) are
defined and given their appropriate parameters (i.e. what species they
will work on, how many levels the bisection has, etc).  Then, in the
algorithms section these moves are called.  This gives you flexibility
over when the moves happen during the PIMC run.  

\begin{itemize}
   \item \texttt{type}:  Type of move that is being
   defined.  For example, if you are defining a BisectionBlock move
   then your type will be BisectionBlock.

   \item \texttt{name}:  The name of your move. This may be anything
   but must be unique for each move in your Input File (the code should
   really check for this, btw).  This is the name you will use in the
   algorithm in specifying that this move is to be called.

\end{itemize}

\subsection{BisectionBlock Moves}
 
\begin{itemize}

  \item \texttt{NumLevels}: integer that is the number of levels the
  bisection should use.
   
  \item \texttt{ActiveSpecies}: Should be deprecated. Please remove
  from your input file. 

  \item \texttt{StepsPerBlock}: integer representing how many steps
  the BisectionBlock takes.  All steps of the bisectionblock are done
  with the same subset of time slices.  When permutations are on, doing $n$
  \texttt{StepsPerBlock} is more effecient then running the move $n$
  times, because it takes advantages of previous computation.  

  
   \item \texttt{Species}: string variable that contains the name of
   the species that the Move operates on.  This name should correspond
   to the value given to \texttt{Name} in the \texttt{Species}
   section.


   \item \texttt{PermuteType}: Indicates how (and whether) this move
   does permutations.  Note that this currently is a better indicator
   of whether the system is acting Bosonic then the statistics type of
   the species.  Choices include \texttt{Table} which performs up to 4
   particle permutations and \texttt{None} which indicates no
   permutations will happen.

   \item \texttt{Array<double,1> Gamma(4)}: Array of enhancement
     factors for 1,2,3, and 4 particle
     permutations when using a \texttt{PermuteType} of \texttt{Table}.
     These {\em must} be greater then 1.  Each factor will
     multiplicatively enhance the
     probability a permutation (of the appropriate size) will be chosen.
     
   \item \texttt{epsilon}: a cutoff that forbids sampling permutations with
     a probability (as determined by the kinetic action) smaller then epsilon 
     of being accepted.  Only to be used with \texttt{PermuteType} of
     \texttt{Table}. Epsilon significantly increases the 
     effeciency of the permutation algorithm moving it from
     approximately a $n^4$ to a $n^2$ algorithm. 

\end{itemize}

\subsection{Open End Moves}
The open end move works by choosing at random one of the two open ends
of the open loops and displacing it randomly within a box (currently
the size of the box is some percent of the total box size. For certain
applications this will not be effective).  It then rebuilds a brownian
bridge between the new ``end'' and the time slice  $2^{\texttt{NumLevels}}$ before the end  using the bisection algorithm.   It is currently IMPORTANT that this move be commented out (both in the moves section and the algorithm) if the code is running in non-OpenLoops mode. Otherwise illegal memory access will happen (we should edit the code to enforce this constraint).
\begin{itemize}
\item The \texttt{type} for this move must be OpenEnd
\item \texttt{NumLevels} This variable designates how many levels are used in the bisection used to rebuild a brownian bridge to the open time slice.
\end{itemize}

\subsection{ShiftMove}
All code in PIMC++ follows the invariant that the initial and final
time slice in memory are not to be altered.   Notice, without time
slice parallelization these two time slices will be permutations of
one another. With time slice parallelization, the last time slice of
processor i will be a permutation of the first time slice of processor
i+1 (modulo the number of processors).  Keeping this invariant is
crucial to the proper functioning of time slice
parallelization. Because of this, shift moves are occassionally
done. A shift move, shifts all of the time slices in memory a random
amount. This allows time slices that used to be unmovable (because of
the above invariant) to be moved.  Without doing this, we would not be
able to ensure ergodicity for the entire path.   
\begin{itemize}
\item The \texttt{type} for this move must be ShiftMove.
\end{itemize}




\section{The ``Observables'' section}
All observables have the following variables:
\begin{itemize}
\item \texttt{type}: The type of the observable is unique for each
  type of observable.  It indicates to the code what observable is being
  defined and specified below in the individual sections.

\item \texttt{Name}: The name of your observable. This may be anything
   but must be unique for each observable in your Input File (the code should
   really check for this, btw).  This is the name you will use in the
   algorithm in specifying that this observable to be called.

\item \texttt{freq}:  The observable will accumulate its statistics
  after being called ``freq'' number of times.  
\item \texttt{dumpFreq}: The observable will write data to the output
  file after being called ``dumpFreq'' number of times. Soon to be
  deprecated.
\item \texttt {Description}: Variable to add comments about your
  observable. Information will be stored in the output file but not
  specifically used for anything.

\end{itemize}
\subsection{Time Analysis}
There are internal clocks in PIMC++ that measure that amount of time
spent in every move and observable.  The time analysis observable
summarizes this information and outputs it to the hdf5 file. It is
recommended that all runs include a time analysis observable (by doing
so, the algorithm can be tuned to spend a reasonable amount of time in
each section of the code) 
The input parameters for the time analysis observables are 
\begin{itemize}
\item The \texttt{type} of this observable must be ``TimeAnalysis''
\end{itemize}
\subsection{Energy}
The energy observable calculates the Kinetic, Short Range, Long Range,
Potential, and Nodal energies. Note that the bulk of this calculation
is done by querying the ActionClasses. At some point in the near
future, it wil be able to be specified what energies are used in the
observable energy but this feature is currently not implemented.  (PUT
HERE THE FORMULAS FOR THESE DIFFERENT ENERGIES) 
The input parameters for the Energy observable are 
\begin{itemize}
\item The \texttt{type} of this observable must be ``Energy''
\end{itemize}

\subsection{Pair Correlation}
The pair correlation observable calculates the pair correlation   
$$g(r)=\frac{3}{4\Pi}\frac{\Omega}{N1*N2}$$ where $N1$ and $N2$ are
the total number of particles in each species. In the case where
species 1 and species 2 are the same, $N2=N1-1$ 
\begin{itemize}
\item The \texttt{type} of this observable must be
`''PairCorrelation''
\item \texttt{Species1}: String variable indication one of the two species used to calculate the
  pair correlation. Must be consistently the same string as used in
  other parts of the code.
\item \texttt{Species2}: String variable indication one of the two species used to calculate the
  pair correlation. Must be consistently the same string as used in
  other parts of the code.
\end{itemize}
The pair correlation observable has a grid subsection. This means that
your pair correlation observable will look like
\rule{0.6cm}{0cm}\shadowbox{\begin{minipage}{4.4in}
\begin{verbatim}
 Section (Observable)
    {
      string type = "PairCorrelation";
      string Name = "HeHePC";
      string Species1 = "He";
      string Species2 = "He";
      string Description="Helium-Helium Pair Correlation";
      int freq=2;
      int dumpFreq=20;
      Section (Grid)
        {
          string type = "Linear";
          double start = 0.0;
          int NumPoints = 100;
        }
    }
\end{verbatim}
\end{minipage}}
The grid section allows for the following variables:
\begin{itemize}
\item \texttt{type}: string variable specifies whether the grid is
  ``Linear'' or of some other type. I've never tested a non-linear
  grid so don't guarentee it works with this observable correctly.
\item \texttt{start}: double variable specifying what gridpoint the grid
  starts at. Typically should be 0.0.
\item \texttt{end}: double variable specificying what gridpoint the
  grid ends at. Will default to something reasonable if omitted.
\item \texttt{NumPoints}: Indicates how many points are used in the grid.
\end{itemize}
\subsection{$n(r)$}
The $n(r)$ observable is an off diagonal observable used for
establishing condensate fractions, the existence of Off Diagonal Long
Range Order, etc. It will only work property when the \texttt{OpenLoops}
variable is set to true.
\begin{itemize}
\item The \texttt{type} of this observable must be ``nofr''
\item \texttt{Species1}: String variable indication one of the two species used to calculate the
  pair correlation. Must be consistently the same string as used in
  other parts of the code.
\item \texttt{Species2}: String variable indication one of the two species used to calculate the
  pair correlation. Must be consistently the same string as used in
  other parts of the code.
\end{itemize}
Like the pair correlation observable, the $n(r)$ observable has a
grid. The parameters for the grid are identical (as follows):
\begin{itemize}
\item \texttt{type}: string variable specifies whether the grid is
  ``Linear'' or of some other type. I've never tested a non-linear
  grid so don't guarentee it works with this observable correctly.
\item \texttt{start}: double variable specifying what gridpoint the grid
  starts at. Typically should be 0.0.
\item \texttt{end}: double variable specificying what gridpoint the
  grid ends at. Will default to something reasonable if omitted.
\item \texttt{NumPoints}: Indicates how many points are used in the grid.
\end{itemize}
\subsection{PathDump}
The PathDump observable outputs the the current state of the path
including the cooridinates of all particles at all points in time as
well as the current permutation.  
\begin{itemize}
\item The \texttt{type} of this observable must be ``PathDump''
\end{itemize}
\section{The ``Algorithm'' section}
In the algorithm section, the specific procedure to be followed in the
PIMC++ run is specified.  There are three basic constructs in this
section: the loop, the move, and the observe. The overlying structure
is a set of loops that are iterated a specified number of times.
Inside these loops, moves and observables are called.  The use of
these constructs are clear from the following example:
\rule{0.6cm}{0cm}\shadowbox{\begin{minipage}{4.4in}
\begin{verbatim}

 Section (Loop){ 
    int steps=20; //This loop will run 20 times
    Section (Move) {string name="BisectHelium";} //Each time this
                                                 // move is called.
    Section (Loop){
      int steps=500;
      Section (Move) {string name="OpenEndMove";}
    }
    Section (Move) {string name = "Shift";}
    Section (Observe) {string name = "TimeObservable"; }
    Section (Observe) {string name = ``HeHePairCorrelation"; }
  }

\end{verbatim}
\end{minipage}}
This piece of code would run 20 times.  It first runs the move that
has the variable name BisectHelium. It then runs the move that has the
variable name OpenEndMove 500 times in another loop.  Finally it runs
the move named shift and calls two observables named TimeObservable
and HeHePairCorrelation. Notice, that calling these observable may do
nothing substational if they haven't been called some modulo or freq
or dumpfreq times.
\chapter{Analysis Tools}
\section{MakeReport}
\section{pathvis++}


\chapter{Modifing pimc++}
\section{Useful things to know}
Each particle in PIMC++ is indexed by a unique integer. 
\section{Adding a new ``Move''}
There are two steps to adding a new move. First, the move needs to be
created and then the appropriate files need to be changed to allow the
move to be ``glued'' into the rest of the code.

There are two types of moves. There are normal moves and then
multistage moves. We will start by discussion normal moves.


All normal moves must inherit from the class ParticleMoveClass and 
create the following four functions: \\
a constructor,\\
\texttt{MakeMove()},\\
\texttt{Read(IOSectionClass \&in)},  \\
texttt{WriteRatio()}.


The constructor of a move must look like the following:\\

\rule{0.6cm}{0cm}\shadowbox{\begin{minipage}{4.4in}
\begin{verbatim}
  NameOfMyMoveClass(PathDataClass &myPathData, 
  IOSectionClass outSection) :
  ParticleMoveClass(myPathData,outSection)
    {
      /* Add code you want here */
    }
\end{verbatim}
\end{minipage}}

It is important to remember to send a pathdata reference and an
  IOSectionClass to ParticleMoveClass as is done above. This is so the
  move has access to the pathdata and an ability to write out data
  like its acceptance ratio

<<<<<<< .mine
The function \texttt{MakeMove()} will be called by the algorithm 
and is in charge of the entire move. This means that it is in charge
of performing the metropolis proceudure by itself (This includes calling
the appropriate actions and then using a random number to decide on 
acceptance or rejection)
At the end of the move, it is {\em mandatory} that \texttt{MakeMove()} 
either accepts or
rejects the move by calling either \\
\rule{0.6cm}{0cm}\shadowbox{\begin{minipage}{4.4in}
\begin{verbatim}
PathData.Path.AcceptMove(startSlice,endSlice,ActiveParticles)
\end{verbatim}
\end{minipage}}\\
if the move is accepted or by calling\\
\rule{0.6cm}{0cm}\shadowbox{\begin{minipage}{4.4in}
\begin{verbatim}
PathData.Path.RejectMove(startSlice,endSlice,ActiveParticles) 
\end{verbatim}
\end{minipage}}\\
 if the move is rejected.  
The parameters sent to these functions
include the range (inclusive) of slices that the move has altered as
well as an array of particle indices 
(of type  \texttt{Array<int,1>} ) that the move has changed.  
By accepting the move, all knowledge of the previous state of the path
will be erased. By rejecting the move all changes the move made in the
slice ranges defined by \texttt{[startSlice:endSlice]} on the particles defined
by the array \texttt{ActiveParticles} will be overwritten by the previous state
of the path at these locations.  


We will not worry about \texttt{WriteRatio()} for the time being other then to
say that it is relevant for getting the acceptance ratio outputs
correctly.

The function \texttt{Read(IOSectionClass \&in)} is called immediately after the
move is created so that the move has the opportunity to read relevant
input information.  When it is called, the IOSectionClass object
has the name of your move which
will be inside \texttt{Section (Moves)}.
 The move is then responsible for reading the input parameters it needs
by calling \texttt{in.Readvar(``varString'',var)}. For example, it might read the 
number of levels it is to operate on by calling 
\texttt{in.ReadVar(``NumLevels'',numLevels)} Note,

opens, but {\em NOT} for closing the section it is in (i.e. there needs to
be exactly the same number of \texttt{OpenSection} and 
\texttt{CloseSection} calls in
your \texttt{Read} function).

To glue the read function into your code, there are currently three
steps. First, perform the appropriate incanation on the configuration scripts.
Secondly,  your move needs
to be added to the .h file of \texttt{MovesClass.h}. This ensures that it will
get included when other files \#include the moves. Finally, inside
PIMCClass.cc, there is a ReadMoves() function which will need to have your
move added to it.  In the series of ``if'' statements, you need to add
\rule{0.6cm}{0cm}\shadowbox{\begin{minipage}{4.4in}
\begin{verbatim}
 else if (MoveType==''MyMoveType''){
    moveName=''MyMoveName'';
    Moves(counter)=new MyMoveClass(PathData,OutFile);
    Moves(counter)->Read(in);
}
\end{verbatim}
\end{minipage}}\\
The moveType will have to be the type the user must type under
\texttt{Section (Move)} for your move.





\section{Adding a new ``Observable''}
There are two steps to adding a new observable. First, the observable
needs to be created and then appropriate files need to be changed to
allow the observable to be ``glued'' into the rest of the code.
\subsection{Creating an observable}
All observables are located in the Observables directory underneath
the src directory. All observables inherit from the function
ObservableClass (located in ObservableBase.h). Every observable has a
``name'' and a ``description'' and all
observables must implement the functions: \\
\texttt{void Accumulate()} \\
\texttt{void WriteBlock()} \\ 
The following functions may be overwritten by an observable, but need
not be: \\
\texttt{void Read(IOSectionClass\& IO)} \\
\texttt{void WriteInfo()} \\
A standard observable should also have two variables called \texttt{Freq} and
\texttt{DumpFreq} that are read in from the input file and respectively
establish how many times the observable must be called before it
accumulates its running average and how many times it must be called
before it outputs its running average to the output file.


To simplify the writing of observables, one should use 
the children of the \texttt{ObservableVar} class. (Such children include 
\texttt{ObservableDouble},\texttt{ObservableVecDouble1},
\texttt{ObsservableInt},etc.) One of these should be used for each
variable you write (recurringly) to your hdf file.  
To use them you must call their constructor with three parameters:
the name of the variable to be used in the output file, a reference to 
an IOSectionClass that will NOT be deleted upon exiting the constructor
(concretely this means that you need to call it with the capatilized and not
lowercase version of IOSection), and I reference to a communicator that 
will be used among the time slice parallel versions of the ...

First we examine how the ObservableVariables are created.


To simplify the writing of observables, one should use 
the children of the \texttt{ObservableVar} class. (Such children include 
\texttt{ObservableDouble},\texttt{ObservableVecDouble1},
\texttt{ObsservableInt},etc.) One of these should be used for each
variable you write (recurringly) to your hdf file.  
To use them you must call their constructor with three parameters:
the name of the variable to be used in the output file, a reference to 
an IOSectionClass that will NOT be deleted upon exiting the constructor
(concretely this means that you need to call it with the capatilized and not
lowercase version of IOSection), and I reference to a communicator that 
will be used among the time slice parallel versions of the ...


\subsubsection{Accumulate()}
The accumulate function gets called each time that an observable is
told to accumulate in the code's algorithm.  A correctly implemented
accumulate should call WriteBlock() every DumpFreq amount of times and
should perform the functions inside of itself every Freq amount of
times it is called.  The functions inside of itself should add onto
its running average the value of the observable it is in charge
of. For example if it was an energy observable, it might accumulate
the running sum of the kinetic and potential energies.


\chapter{Code Verification}
As with any computational physics code, it is important to
authenticate that the code produces accurate results. We do that by
verifying the results of calculations against analytical results,
other Path Integral codes (such as UPI) and general sanity checks. In
this chapter we document many of these tests to increase confidence in
the code's use.  We encourage other quantum monte carlo codes to
perform a similar exercise (ideally with similar systems to allow
comparison among different codes.


\end{document}
