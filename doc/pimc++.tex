\documentclass{book}
\author{Kenneth P. Esler, Jr.}
\title{pimc++}
%\subtitle{Path Integral Monte Carlo Simulation Tool}

\begin{document}
\maketitle
\chapter{Introduction}
{\em pimc++} is a general, high-performance tool to perform
fully-correlated simulations of quantum systems at finite temperature.
In essense, it samples the many-body thermal density matrix to compute
thermal equilibrium averages of quantum observables.  Written in very
object-oriented C++, it is intended to be a nearly universal tool that
can be easily extended to perform novel simulations.  Its current
features include:
\begin{itemize}
  \item Arbitrary number of particles, species, and timeslices,
        limited only by memory requirements.
  \item Time-slices can be distributed across processors using MPI to
        communicate, making possible efficient simulations with
        thousands of time-slices.
  \item Multiple ``clones'' of the same simulation may be run with
        automatic generation of different random number seeds.  This
        can be used to improve statistics.
  \item Very high-accuracy pair-actions can be used.  These are
        generated with the coupled squarer++/fitter++ package.
  \item Bose, Fermi, and Boltzmann statistics are supported.
  \item Output is efficiently and accurately stored in HDF5 format.
  \item Extremely flexible input file format allows specification of
        the Monte Carlo algorithm in the input.
  \item Free and periodic boundary conditions are specified
        independently for each dimension allowing slab and column
        geometries. 
  \item Statistical analysis and HTML report generation with
        automatic creation of embedded plots.
  \item 3D OpenGL-based visualization tool can be used for debugging
        and pedagogy.  Supports exporting to the POVray ray-tracer for 
        photo-realistic rendering.  Also supports MPEG4 movie generation.
\end{itemize}

\chapter{Installing}
Suffice it to say that installing {\em pimc++} can be a quite a pain
in the caboose.

\chapter{Input Files}
\section{Introduction}
\section{Defining variables}
\section{Creating hierachy with sections}

\section{Format}
\section{Global parameters}
\section{The ``System'' section}
\section{The ``Action'' section}
\section{The ``Moves'' section}
\section{The ``Observables'' section}
\section{The ``Algorithm'' section}

\chapter{Analysis Tools}
\section{MakeReport}
\section{pathvis++}

\chapter{Modifing pimc++}
\section{Adding a new ``Move''}
\section{Adding a new ``Observable''}

\end{document}
